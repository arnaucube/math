\documentclass{article}
\usepackage[utf8]{inputenc}
\usepackage{amsfonts}
\usepackage{amsthm}
\usepackage{amsmath}
\usepackage{amssymb}
\usepackage{mathtools}
\usepackage{enumerate}
\usepackage{hyperref}
\hypersetup{
    colorlinks,
    citecolor=black,
    filecolor=black,
    linkcolor=black,
    urlcolor=blue
}
% \usepackage{xcolor}

% prevent warnings of underfull \hbox:
% \usepackage{etoolbox}
% \apptocmd{\sloppy}{\hbadness 4000\relax}{}{}

\theoremstyle{definition}
\newtheorem{definition}{Def}[section]
\newtheorem{theorem}[definition]{Thm}
\newtheorem{innersolution}{}
\newenvironment{solution}[1]
{\renewcommand\theinnersolution{#1}\innersolution}
{\endinnersolution}


\title{Weil Pairing - study}
\author{arnaucube}
\date{August 2022}

\begin{document}
\maketitle

\begin{abstract}
  Notes taken from \href{https://sites.google.com/site/matanprasma/artifact}{Matan Prsma} math seminars and also while reading about Bilinear Pairings. Usually while reading papers and books I take handwritten notes, this document contains some of them re-written to $LaTeX$.

	The notes are not complete, don't include all the steps neither all the proofs. I use these notes to revisit the concepts after some time of reading the topic.
\end{abstract}

\tableofcontents

\section{Rational functions}

Let $E/\Bbbk$ be an elliptic curve defined by: $y^2 = x^3 + Ax + B$.

\paragraph{set of polynomials over $E$:}
$\Bbbk[E] := \Bbbk[x,y] / (y^2 - x^3 - Ax - B =0)$

we can replace $y^2$ in the polynomial $f \in \Bbbk[E]$ with $x^3 + Ax + B$

\paragraph{canonical form:} $f(x,y) = v(x)+y w(x)$ for $v, w \in \Bbbk[x]$
\paragraph{conjugate:} $\overline{f} = v(x) - y w(x)$
\paragraph{norm:} $N_f = f \cdot \overline{f} = v(x)^2 - y^2 w(x)^2 = v(x)^2 - (x^3 + Ax + B) w(x)^2 \in \Bbbk[x] \subset \Bbbk[E]$

we can see that $N_{fg} = N_f \cdot N_g$

\paragraph{set of rational functions over $E$:}
$\Bbbk(E) := \Bbbk[E] \times \Bbbk[E]/ \thicksim$

For $r\in \Bbbk(E)$ and a finite point $P \in E(\Bbbk)$, $r$ is \emph{finite} at $P$ iff
$$\exists~ r=\frac{f}{g} ~\text{with}~ f,g \in \Bbbk[E],~ s.t.~ g(P) \neq 0$$
We define $r(P)=\frac{f(P)}{g(P)}$. Otherwise, $r(P)=\infty$.

Remark: $r=\frac{f}{g} \in \Bbbk(E)$, $r=\frac{f}{g}=\frac{f \cdot \overline{g}}{g \cdot \overline{g}} = \frac{f \overline{g}}{N_g}$, thus
$$r(x,y)=\frac{ (f \overline{g})(x,y)}{N_g(x,y)} = \underbrace{ \frac{v(x)}{N_g(x)} + y \frac{w(x)}{N_g(x)} }_\text{canonical form of $r(x,y)$}$$

\paragraph{degree of $f$:} Let $f\in \Bbbk[E]$, in canonical form: $f(x,y) = v(x) + y w(x)$,
$$deg(f) := max\{ 2 \cdot deg_x(v), 3+2 \cdot deg_x(w) \}$$

For $f,g \in \Bbbk[E]$:
\begin{enumerate}[i.]
  \item $deg(f) = deg_x(N_f)$
  \item $deg(f \cdot g) = deg(f) + deg(g)$
\end{enumerate}

\begin{definition}
  Let $r=\frac{f}{g} \in \Bbbk(E)$
  \begin{enumerate}[i.]
    \item if $deg(f) < deg(g):~ r(0)=0$
    \item if $deg(f) > deg(g):~ r ~\text{is not finite at}~ 0$
    \item if $deg(f) = deg(g)$ with $deg(f)$ even:\\
      $f$'s canonical form leading terms $ax^d$\\
      $g$'s canonical form leading terms $bx^d$\\
      $a,b \in \Bbbk^\times,~ d=\frac{deg(f)}{2}$, set $r(0)=\frac{a}{b}$
    \item if $deg(f) = deg(g)$ with $deg(f)$ odd\\
      $f$'s canonical form leading terms $ax^d$\\
      $g$'s canonical form leading terms $bx^d$\\
      $a,b \in \Bbbk^\times,~ deg(f)=deg(g)=3+2d$, set $r(0)=\frac{a}{b}$
  \end{enumerate}
\end{definition}

\subsection{Zeros, poles, uniformizers and multiplicities}

$r \in \Bbbk(E)$ has a \emph{zero} in $P\in E$ if $r(P)=0$\\
$r \in \Bbbk(E)$ has a \emph{pole} in $P\in E$ if $r(P)$ is not finite.

\paragraph{uniformizer:} Let $P\in E$,
uniformizer: rational function $u \in \Bbbk(E)$ with $u(P)=0$ if
$\forall r\in \Bbbk(E) \setminus \{0\},~ \exists d \in \mathbb{Z},~ s\in \Bbbk(E)$ finite at $P$ with $s(P) \neq 0$ s.t.
$$r=u^d \cdot s$$

\paragraph{order:} Let $P \in E(\Bbbk)$, let $u \in \Bbbk(E)$ be a uniformizer at $P$.
For $r \in \Bbbk(E) \setminus \{0\}$ being a rational function with $r=u^d \cdot s$ with $s(P)\neq 0, \infty$, we say that $r$ has \emph{order} $d$ at $P$ ($ord_P(r)=d$).

\paragraph{multiplicity:} \emph{multiplicity of a zero} of $r$ is the order of $r$ at that point, \emph{multiplicity of a pole} of $r$ is the order of $r$ at that point.

if $P \in E(\Bbbk)$ is neither a zero or pole of $r$, then $ord_P(r)=0$ ($=d,~ r=u^0s$).

\vspace{0.5cm}
\begin{minipage}{4.3 in}
    \paragraph{Multiplicities, from the book "Elliptic Tales"} (p.69), to provide intuition

    Factorization into \emph{linear factors}: $p(x)=c\cdot (x-a_1) \cdots (x-a_d)$\\
    $d$: degree of $p(x)$, $a_i \in \Bbbk$\\
    Solutions to $p(x)=0$ are $x=a_1, \ldots, a_d$ (some $a_i$ can be repeated)\\
    eg.: $p(x)=(x-1)(x-1)(x-3)$, solutions to $p(x)=0:~ 1, 1, 3$\\
    $x=1$ is a solution to $p(x)=0$ of \emph{multiplicity} 2.

    The total number of solutions (counted with multiplicity) is $d$, the degree of the polynomial whose roots we are finding.
\end{minipage}


\section{Divisors}

\begin{definition}{Divisor}
  $$D= \sum_{P \in E(\Bbbk)} n_p \cdot [P]$$
\end{definition}

\begin{definition}{Degree \& Sum}
  $$deg(D)= \sum_{P \in E(\Bbbk)} n_p$$
  $$sum(D)= \sum_{P \in E(\Bbbk)} n_p \cdot P$$
\end{definition}


The set of all divisors on $E$ forms a group: for $D = \sum_{P\in E(\Bbbk)} n_P[P]$ and $D' = \sum_{P\in E(\Bbbk)} m_P[P]$,
$$D+D' = \sum_{P\in E(\Bbbk)} (n_P + m_P)[P]$$

\begin{definition}{Associated divisor}
  $$div(r) = \sum_{P \in E(\Bbbk)} ord_P(r)[P]$$
\end{definition}

Observe that
\begin{enumerate}
  \item[] $div(rs) = div(r)+div(s)$
  \item[] $div(\frac{r}{s}) = div(r)-div(s)$
\end{enumerate}

Observe that
$$\sum{P \in E(\Bbbk)} ord_P(r) \cdot P = 0$$

\begin{definition}{Support}
  $$\sum_P n_P[P], ~\forall P \in E(\Bbbk) \mid n_P \neq 0$$
\end{definition}

\begin{definition}{Principal divisor}
  iff $deg(D)=0$ and $sum(D)=0$
\end{definition}
$D \sim D'$ iff $D - D'$ is principal.


\begin{definition}{Evaluation of a rational function} (function $r$ evaluated at $D$)
  $$r(D)= \prod r(P)^{n_p}$$
\end{definition}

\section{Weil reciprocity}
\begin{theorem}{(Weil reciprocity)}
  Let $E/ \Bbbk$ be an e.c. over an alg. closed field. If $r,~s \in \Bbbk\setminus \{0\}$ are rational functions whose divisors have disjoint support, then
$$r(div(s)) = s(div(r))$$
\end{theorem}
Proof. (todo)

\section{Generic Weil Pairing}
Let $E(\Bbbk)$, with $\Bbbk$ of char $p$, $n$ s.t. $p \nmid n$.

$\Bbbk$ large enough: $E(\Bbbk)[n] = E(\overline{\Bbbk}) = \mathbb{Z}_n \oplus \mathbb{Z}_n$ (with $n^2$ elements).

For $P, Q \in E[n]$,
\begin{align*}
  D_P &\sim [P] - [0]\\
  D_Q &\sim [Q] - [0]
\end{align*}

We need them to have disjoint support:
\begin{align*}
  D_P &\sim [P] - [0]\\
  D_Q' &\sim [Q+T] - [T]
\end{align*}

$$\Delta D = D_Q - D_Q' = [Q] - [0] - [Q+T] + [T]$$


Note that $n D_P$ and $n D_Q$ are principal. Proof:
\begin{align*}
  n D_P &= n [P] - n [O]\\
  deg(n D_P) &= n - n = 0\\
  sum(n D_P) &= nP - nO = 0
\end{align*}
($nP = 0$ bcs. $P$ is n-torsion)

Since $n D_P,~ n D_Q$ are principal, we know that $f_P,~ f_Q$ exist.

Take
\begin{align*}
  f_P &: div(f_P) = n D_P\\
  f_Q &: div(f_Q) = n D_Q
\end{align*}

We define
$$
e_n(P, Q) = \frac{f_P(D_Q)}{f_Q(D_P)}
$$

Remind: evaluation of a rational function over a divisor $D$:
\begin{align*}
  D &= \sum n_P [P]\\
  r(D) &= \prod r(P)^{n_P}
\end{align*}

If $D_P = [P+S] - [S],~~ D_Q=[Q-T]-[T]$ what is $e_n(P, Q)$?

\begin{align*}
  f_P(D_Q) &= f_P(Q+T)^1 \cdot f_P(T)^{-1}\\
  f_Q(D_P) &= f_Q(P+S)^1 \cdot f_Q(S)^{-1}
\end{align*}

$$
e_n(P, Q) = \frac{f_P(Q+T)}{f_P(T)} / \frac{f_Q(P+S)}{f_Q(S)}
$$

with $S \neq \{O, P, -Q, P-Q \}$.


\section{Properties}


\section{Exercises}
\emph{An Introduction to Mathematical Cryptography, 2nd Edition} - Section 6.8. Bilinear pairings on elliptic curves

\begin{solution}{6.29}
  $div(R(x) \cdot S(x)) = div( R(x)) + div( S(x))$, where $R(x), S(x)$ are rational functions.
  \\proof:\\
  \emph{Norm} of $f$: $N_f = f \cdot \overline{f}$, and we know that $N_{fg} = N_f \cdot N_g~\forall~\Bbbk[E]$,\\
  then $$deg(f) = deg_x(N_f)$$\\
  and $$deg(f \cdot g) = deg(f) + deg(g)$$

  Proof:
  $$deg(f \cdot g) = deg_x(N_{fg}) = deg_x(N_f \cdot N_g)$$
  $$= deg_x(N_f) + deg_x(N_g) = deg(f) + deg(g)$$

  So, $\forall P \in E(\Bbbk),~ ord_P(rs) = ord_P(r) + ord_P(s)$.\\
  As $div(r) = \sum_{P\in E(\Bbbk)} ord_P(r)[P]$, $div(s) = \sum ord_P(s)[P]$.

  So,
  $$div(rs) = \sum ord_P(rs)[P]$$
  $$= \sum ord_P(r)[P] + \sum ord_P(s)[P] = div(r) + div(s)$$
\end{solution}

\vspace{0.5cm}

\begin{solution}{6.31}
  $$e_m(P, Q) = e_m(Q, P)^{-1} \forall P, Q \in E[m]$$
  Proof:
  We know that $e_m(P, P) = 1$, so:
  $$1 = e_m(P+Q, P+Q) = e_m(P, P) \cdot e_m(P, Q) \cdot e_m(Q, P) \cdot e_m(Q, Q)$$

  and we know that $e_m(P, P) = 1$, then we have:
  $$1 = e_m(P, Q) \cdot e_m(Q, P)$$
  $$\Longrightarrow e_m(P, Q) = e_m(Q, P)^{-1}$$
\end{solution}




\end{document}
