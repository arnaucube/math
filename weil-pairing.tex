\documentclass{article}
\usepackage[utf8]{inputenc}
\usepackage{amsfonts}
\usepackage{amsthm}
\usepackage{amsmath}
\usepackage{amssymb}
\usepackage{enumerate}
\usepackage{hyperref}
\hypersetup{
    colorlinks,
    citecolor=black,
    filecolor=black,
    linkcolor=black,
    urlcolor=blue
}
% \usepackage{xcolor}

% prevent warnings of underfull \hbox:
% \usepackage{etoolbox}
% \apptocmd{\sloppy}{\hbadness 4000\relax}{}{}

\theoremstyle{definition}
\newtheorem{definition}{Def}[section]
\newtheorem{theorem}[definition]{Thm}
\newtheorem{innersolution}{}
\newenvironment{solution}[1]
{\renewcommand\theinnersolution{#1}\innersolution}
{\endinnersolution}


\title{Bilinear Pairings - study}
\author{arnaucube}
\date{August 2022}

\begin{document}

\maketitle

\begin{abstract}
  Notes taken from \href{https://sites.google.com/site/matanprasma/artifact}{Matan Prsma} math seminars and also while reading about Bilinear Pairings. Usually while reading papers and books I take handwritten notes, this document contains some of them re-written to $LaTeX$.

	The notes are not complete, don't include all the steps neither all the proofs. I use these notes to revisit the concepts after some time of reading the topic.
\end{abstract}

\tableofcontents

\section{Weil reciprocity}

\section{Generic Weil Pairing}

\begin{definition}{Divisor}
  $$D= \sum_{P \in E(\mathbb{K})} n_p \cdot [P]$$
\end{definition}

\begin{definition}{Degree \& Sum}
  $$deg(D)= \sum_{P \in E(\mathbb{K})} n_p$$
  $$sum(D)= \sum_{P \in E(\mathbb{K})} n_p \cdot P$$
\end{definition}

\begin{definition}{Principal divisor}
  iff $deg(D)=0$ and $sum(D)=0$
\end{definition}
$D \sim D'$ iff $D - D'$ is principal.


\begin{definition}{Evaluation of a rational function}
  $$r(D)= \prod r(P)^{n_p}$$
\end{definition}

\subsection{Generic Weil Pairing}
Let $E(\mathbb{K})$, with $\mathbb{K}$ of char $p$, $n$ s.t. $p \nmid n$.

$\mathbb{K}$ large enough: $E(\mathbb{K})[n] = E(\mathbb{\overline{K}}) = \mathbb{Z}_n \oplus \mathbb{Z}_n$ (with $n^2$ elements).

$P, Q \in E[n]$:
$$D_P \sim [P] - [0]$$
$$D_Q \sim [Q] - [0]$$
We need them to have disjoint support:
$$D_P \sim [P] - [0]$$
$$D_Q \sim [Q+T] - [T]$$

$$\Delta D = D_Q - D_Q' = [Q] - [0] - [Q+T] + [T]$$


\section{Exercises}
\emph{An Introduction to Mathematical Cryptography, 2nd Edition} - Section 6.8. Bilinear pairings on elliptic curves

\begin{solution}{6.29}
  $div(R(x) \cdot S(x)) = div( R(x)) + div( S(x))$, where $R(x), S(x)$ are rational functions.
  \\proof:\\
  \emph{Norm} of $f$: $N_f = f \cdot \overline{f}$, and we know that $N_{fg} = N_f \cdot N_g~\forall~\mathbb{K}[E]$,\\
  then $$deg(f) = deg_x(N_f)$$\\
  and $$deg(f \cdot g) = deg(f) + deg(g)$$

  Proof:
  $$deg(f \cdot g) = deg_x(N_{fg}) = deg_x(N_f \cdot N_g)$$
  $$= deg_x(N_f) + deg_x(N_g) = deg(f) + deg(g)$$

  So, $\forall P \in E(\mathbb{K}),~ ord_P(rs) = ord_P(r) + ord_P(s)$.\\
  As $div(r) = \sum_{P\in E(\mathbb{K})} ord_P(r)[P]$, $div(s) = \sum ord_P(s)[P]$.

  So,
  $$div(rs) = \sum ord_P(rs)[P]$$
  $$= \sum ord_P(r)[P] + \sum ord_P(s)[P] = div(r) + div(s)$$
\end{solution}

\vspace{0.5cm}

\begin{solution}{6.31}
  $$e_m(P, Q) = e_m(Q, P)^{-1} \forall P, Q \in E[m]$$
  Proof:
  We know that $e_m(P, P) = 1$, so:
  $$1 = e_m(P+Q, P+Q) = e_m(P, P) \cdot e_m(P, Q) \cdot e_m(Q, P) \cdot e_m(Q, Q)$$

  and we know that $e_m(P, P) = 1$, then we have:
  $$1 = e_m(P, Q) \cdot e_m(Q, P)$$
  $$\Longrightarrow e_m(P, Q) = e_m(Q, P)^{-1}$$
\end{solution}




\end{document}
