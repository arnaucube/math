\documentclass{article}
\usepackage[utf8]{inputenc}
\usepackage{amsfonts}
\usepackage{amsthm}
\usepackage{amsmath}
\usepackage{mathtools}
\usepackage{enumerate}
\usepackage{hyperref}
\usepackage{xcolor}
\usepackage{pgf-umlsd} % diagrams
\usepackage{centernot}
\usepackage{algorithm}
\usepackage{algpseudocode}


% prevent warnings of underfull \hbox:
\usepackage{etoolbox}
\apptocmd{\sloppy}{\hbadness 4000\relax}{}{}

\theoremstyle{definition}
\newtheorem{definition}{Def}[section]
\newtheorem{theorem}[definition]{Thm}

% custom lemma environment to set custom numbers
\newtheorem{innerlemma}{Lemma}
\newenvironment{lemma}[1]
{\renewcommand\theinnerlemma{#1}\innerlemma}
{\endinnerlemma}


\title{Notes on HyperNova}
\author{arnaucube}
\date{May 2023}

\begin{document}

\maketitle

\begin{abstract}
	Notes taken while reading about Spartan \cite{cryptoeprint:2023/573}, \cite{cryptoeprint:2023/552}.

	Usually while reading papers I take handwritten notes, this document contains some of them re-written to $LaTeX$.

	The notes are not complete, don't include all the steps neither all the proofs.
\end{abstract}

\tableofcontents


\section{CCS}
\subsection{R1CS to CCS overview}

\begin{description}
	\item[R1CS instance] $S_{R1CS} = (m, n, N, l, A, B, C)$\\
		where $m, n$ are such that $A \in \mathbb{F}^{m \times n}$, and $l$ such that the public inputs $x \in \mathbb{F}^l$. Also $z=(w, 1, x) \in \mathbb{F}^n$, thus $w \in \mathbb{F}^{n-l-1}$.
	\item[CCS instance] $S_{CCS} = (m, n, N, l, t, q, d, M, S, c)$\\
		where we have the same parameters than in $S_{R1CS}$, but additionally:\\
		$t=|M|$, $q = |c| = |S|$, $d$= max degree in each variable.
	\item[R1CS-to-CCS parameters] $n=n,~ m=m,~ N=N,~ l=l,~ t=3,~ q=2,~ d=2$, $M=\{A,B,C\}$, $S=\{\{0,~1\},~ \{2\}\}$, $c=\{1,-1\}$
\end{description}

The CCS relation check:
$$\sum_{i=0}^{q-1} c_i \cdot \bigcirc_{j \in S_i} M_j \cdot z ==0$$

where $z=(w, 1, x) \in \mathbb{F}^n$.

In our R1CS-to-CCS parameters is equivalent to

\begin{align*}
	&c_0 \cdot ( (M_0 z) \circ (M_1 z) ) + c_1 \cdot (M_2 z) ==0\\
	\Longrightarrow &1 \cdot ( (A z) \circ (B z) ) + (-1) \cdot (C z) ==0\\
	\Longrightarrow &( (A z) \circ (B z) ) - (C z) ==0
\end{align*}

which is equivalent to the R1CS relation: $Az \circ Bz == Cz$

An example of the conversion from R1CS to CCS implemented in SageMath can be found at\\
\href{https://github.com/arnaucube/math/blob/master/r1cs-ccs.sage}{https://github.com/arnaucube/math/blob/master/r1cs-ccs.sage}.

\subsection{Committed CCS}
$R_{CCCS}$ instance: $(C, \mathsf{x})$, where $C$ is a commitment to a multilinear polynomial in $s'-1$ variables.

Sat if:
\begin{enumerate}[i.]
	\item $\text{Commit}(pp, \widetilde{w}) = C$
	\item $\sum_{i=1}^q c_i \cdot \left( \prod_{j \in S_i} \left( \sum_{y \in \{0,1\}^{\log m}} \widetilde{M}_j(x, y) \cdot \widetilde{z}(y) \right) \right)$\\
		where $\widetilde{z}(y) = \widetilde{(w, 1, \mathsf{x})}(x) ~\forall x \in \{0, 1\}^{s'}$
\end{enumerate}


\subsection{Linearized Committed CCS}
$R_{LCCCS}$ instance: $(C, u, \mathsf{x}, r, v_1, \ldots, v_t)$, where $C$ is a commitment to a multilinear polynomial in $s'-1$ variables, and $u \in \mathbb{F},~ \mathsf{x} \in \mathbb{F}^l,~ r \in \mathbb{F}^s,~ v_i \in \mathbb{F} ~\forall i \in [t]$.

Sat if:
\begin{enumerate}[i.]
	\item $\text{Commit}(pp, \widetilde{w}) = C$
	\item $\forall i \in [t],~ v_i = \sum_{y \in \{0,1\}^{s'}} \widetilde{M}_i(r, y) \cdot \widetilde{z}(y)$\\
		where $\widetilde{z}(y) = \widetilde{(w, u, \mathsf{x})}(x) ~\forall x \in \{0, 1\}^{s'}$
\end{enumerate}


\section{Multifolding Scheme for CCS}
Recall sum-check protocol notation: \underline{$C \leftarrow \langle P, V(r) \rangle (g, l, d, T)$}:
$$T=\sum_{x_1 \in \{0,1\}} \sum_{x_2 \in \{0,1\}} \cdots \sum_{x_l \in \{0,1\}} g(x_1, x_2, \ldots, x_l)$$
where $g$ is a $l$-variate polynomial, with degree at most $d$ in each variable, and $T$ is the claimed value.

\vspace{1cm}

Let $s= \log m,~ s'= \log n$.

\begin{enumerate}
	\item $V \rightarrow P: \gamma \in^R \mathbb{F},~ \beta \in^R \mathbb{F}^s$
	\item $V: r_x' \in^R \mathbb{F}^s$
	\item $V \leftrightarrow P$: sum-check protocol:
		$$c \leftarrow \langle P, V(r_x') \rangle (g, s, d+1, \overbrace{\sum_{j \in [t]} \gamma^j \cdot v_j}^\text{T})$$
		where:
		\begin{align*}
			g(x) &:= \left( \sum_{j \in [t]} \gamma^j \cdot L_j(x) \right) + \gamma^{t+1} \cdot Q(x)\\
			\text{for LCCCS:}~ L_j(x) &:= \widetilde{eq}(r_x, x) \cdot \left(
				\underbrace{\sum_{y \in \{0,1\}^{s'}} \widetilde{M}_j(x, y) \cdot \widetilde{z}_1(y)}_\text{this is the check from LCCCS}
			\right)\\
				\text{for CCCS:}~ Q(x) := &\widetilde{eq}(\beta, x) \cdot \left(
				\underbrace{ \sum_{i=1}^q c_i \cdot \prod_{j \in S_i} \left( \sum_{y \in \{0, 1\}^{s'}} \widetilde{M}_j(x, y) \cdot \widetilde{z}_2(y) \right) }_\text{this is the check from CommittedCCS}
			\right)
		\end{align*}
		Notice that $v_j= \sum_{y\in \{0,1\}^{s'}} \widetilde{M}_j(r, y) \cdot \widetilde{z}(y) = \sum_{x\in \{0,1\}^s} L_j(x)$.
	\item $P \rightarrow V$: $\left( (\sigma_1, \ldots, \sigma_t), (\theta_1, \ldots, \theta_t) \right)$, where $\forall j \in [t]$,
		$$\sigma_j = \sum_{y \in \{0,1\}^{s'}} \widetilde{M}_j(r_x', y) \cdot \widetilde{z}_1(y)$$
		$$\theta_j = \sum_{y \in \{0, 1\}^{s'}} \widetilde{M}_j(r_x', y) \cdot \widetilde{z}_2(y)$$
		where $\sigma_j,~\theta_j$ are the checks from LCCCS and CCCS respectively with $x=r_x'$.
	\item V: $e_1 \leftarrow \widetilde{eq}(r_x, r_x')$, $e_2 \leftarrow \widetilde{eq}(\beta, r_x')$\\
		check:
		$$c = \left( \sum_{j \in [t]} \gamma^j e_1 \sigma_j + \gamma^{t+1} e_2 \left( \sum_{i=1}^q c_i \cdot \prod_{j \in S_i} \sigma \right) \right)$$
		which should be equivalent to the $g(x)$ computed by $V,P$ in the sum-check protocol.
	\item $V \rightarrow P: \rho \in^R \mathbb{F}$
	\item $V, P$: output the folded LCCCS instance $(C', u', \mathsf{x}', r_x', v_1', \ldots, v_t')$, where $\forall i \in [t]$:
		\begin{align*}
			C' &\leftarrow C_1 + \rho \cdot C_2\\
			u' &\leftarrow u + \rho \cdot 1\\
			\mathsf{x}' &\leftarrow \mathsf{x}_1 + \rho \cdot \mathsf{x}_2\\
			v_i' &\leftarrow \sigma_i + \rho \cdot \theta_i
		\end{align*}
	\item $P$: output folded witness: $\widetilde{w}' \leftarrow \widetilde{w}_1 + \rho \cdot \widetilde{w}_2$.
\end{enumerate}



%%%%%% APPENDIX
\appendix
\section{Appendix: Some details}
This appendix contains some notes on things that don't specifically appear in the paper, but that would be needed in a practical implementation of the scheme.

\subsection{Matrix and Vector to Sparse Multilinear Extension}

Let $M \in \mathbb{F}^{m \times n}$ be a matrix. We want to compute its MLE
$$\widetilde{M}(x_1, \ldots, x_l) = \sum_{e \in \{0, 1 \}^l} M(e) \cdot \widetilde{eq}(x, e)$$

We can view the matrix $M \in \mathbb{F}^{m \times n}$ as a function with the following signature:
$$M(\cdot): \{0,1\}^s \times \{0,1\}^{s'} \rightarrow \mathbb{F}$$
where $s = \lceil \log m \rceil,~ s' = \lceil \log n \rceil$.

An entry in $M$ can be accessed with a $(s+s')$-bit identifier.

eg.:
$$
M = \begin{pmatrix}
1 & 2 & 3\\
4 & 5 & 6\\
\end{pmatrix}
\in \mathbb{F}^{3 \times 2}
$$

$m = 3,~ n = 2,~~~ s = \lceil \log 3 \rceil = 2,~ s' = \lceil \log 2 \rceil = 1$

So, $M(s_0, s_1) = x$, where $s_0 \in \{0,1\}^s,~ s_1 \in \{0,1\}^{s'},~ x \in \mathbb{F}$

$$
M = \begin{pmatrix}
M(00,0) & M(01,0) & M(10,0)\\
M(00,1) & M(01,1) & M(10,1)\\
\end{pmatrix}
\in \mathbb{F}^{3 \times 2}
$$

This logic can be defined as follows:

\begin{algorithm}[H]
\caption{Generating a Sparse Multilinear Polynomial from a matrix}
\begin{algorithmic}
	\State set empty vector $v \in (\text{index:}~ \mathbb{Z}, x: \mathbb{F})^{s \times s'}$
	\For {$i$ to $n$}
	\For {$j$ to $m$}
		\If {$M_{i,j} \neq 0$}
			\State $v.\text{append}( \{ \text{index}: i \cdot m + j,~ x: M_{i,j} \} )$
		\EndIf
	\EndFor
	\EndFor
	\State return $v$    \Comment {$v$ represents the evaluations of the polynomial}
\end{algorithmic}
\end{algorithm}

Once we have the polynomial, its MLE comes from
$$\widetilde{M}(x_1, \ldots, x_{s+s'}) = \sum_{e \in \{0,1\}^{s+s'}} M(e) \cdot \widetilde{eq}(x, e)$$

$$M(X) \in \mathbb{F}[X_1, \ldots, X_s]$$

\paragraph{Multilinear extensions of vectors}
Given a vector $u \in \mathbb{F}^m$, the polynomial $\widetilde{u}$ is the MLE of $u$, and is obtained by viewing $u$ as a function mapping ($s=\log m$)
$$u(x): \{0,1\}^s \rightarrow \mathbb{F}$$
$\widetilde{u}(x, e)$ is the multilinear extension of the function $u(x)$
$$\widetilde{u}(x_1, \ldots, x_s) = \sum_{e \in \{0,1\}^s} u(e) \cdot \widetilde{eq}(x, e)$$

\bibliography{paper-notes.bib}
\bibliographystyle{unsrt}

\end{document}
