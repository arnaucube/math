\documentclass{article}
\usepackage[utf8]{inputenc}
\usepackage{amsfonts}
\usepackage{amsthm}
\usepackage{amsmath}
\usepackage{enumerate}
\usepackage{hyperref}
\hypersetup{
    colorlinks,
    citecolor=black,
    filecolor=black,
    linkcolor=black,
    urlcolor=blue
}
\usepackage{xcolor}

% prevent warnings of underfull \hbox:
\usepackage{etoolbox}
\apptocmd{\sloppy}{\hbadness 4000\relax}{}{}

\theoremstyle{definition}
\newtheorem{definition}{Def}[section]
\newtheorem{theorem}[definition]{Thm}


\title{Notes on Sonic}
\author{arnaucube}
\date{}

\begin{document}

\maketitle

\begin{abstract}
	Notes taken while reading Sonic paper \cite{cryptoeprint:2019/099}. Usually while reading papers I take handwritten notes, this document contains some of them re-written to $LaTeX$.

	The notes are not complete, don't include all the steps neither all the proofs.
\end{abstract}

\tableofcontents


\section{Sonic}

\subsection{Structured Reference String}
$\{ \{g^{x^i}\}_{i=-d}^d, \{ g^{\alpha x^i} \}_{i=-d, i \neq 0}^d, \{ h^{x^i}, h^{\alpha x^i} \}_{i=-d}^d, e(g, h^\alpha) \}$

\subsection{System of constraints}
Multiplication constraint: $a \cdot b = c$

$Q$ linear constraints:
$$
a \cdot u_q + b \cdot v_q + c \cdot w_q = k_q
$$

with $u_q, v_q, w_q \in \mathbb{F}^n$, and $k_q \in \mathbb{F}_p$.

\vspace{0.5cm}
Example: $x^2 + y^2 = z$

$$a = (x, y), \qquad b = (x, y), \qquad c = (x^2, y^2)$$
\begin{enumerate}[i.]
    \item $(x, y) \cdot (1, 0) + (x, y) \cdot (-1, 0) + (x^2, y^2) \cdot (0, 0) = 0 \longrightarrow x - x = 0$
    \item $(x, y) \cdot (0, 1) + (x, y) \cdot (0, -1) + (x^2, y^2) \cdot (0, 0) = 0 \longrightarrow y - y = 0$
    \item $(x, y) \cdot (0, 0) + (x, y) \cdot (0, 0) + (x^2, y^2) \cdot (1, 1) = z \longrightarrow x^2 + y^2 = z$
\end{enumerate}

So,
$$u_1 = (1, 0) \quad v_1=(-1, 0) \quad w_1=(0, 0) \quad k_1=0$$
$$u_2 = (0, 1) \quad v_2=(0, -1) \quad w_2=(0, 0) \quad k_2=0$$
$$u_3 = (0, 0) \quad v_3=(0, 0) \quad w_3=(1, 1) \quad k_2=z$$

\vspace{1cm}

Compress n multiplication constraints into an equation in formal indeterminate $Y$:
$$\sum_{i=1}^n (a_i b_i - c_i) \cdot Y^i = 0$$
encode into negative exponents of $Y$:
$$\sum_{i=1}^n (a_i b_i - c_i) \cdot Y^-i = 0$$

Also, compress the $Q$ linear constraints, scaling by $Y^n$ to preserve linear independence:
$$
\sum_{q=1}^Q (a \cdot u_q + b \cdot v_q + c \cdot w_q - k_q) \cdot Y^{q+n} = 0
$$

Polys:

\begin{align}
\nonumber & u_i(Y) = \sum_{q=1}^Q Y^{q+n} \cdot u_{q, i}\\
\nonumber & v_i(Y) = \sum_{q=1}^Q Y^{q+n} \cdot v_{q, i}\\
\nonumber & w_i(Y) = -Y^i - Y^{-1} + \sum_{q=1}^Q Y^{q+n} \cdot w_{q, i}\\
\nonumber & k(Y) = \sum_{q=1}^Q Y^{q+n} \cdot k_q
\end{align}

Combine the multiplicative and linear constraints to:

\begin{align}
\nonumber & a \cdot u(Y) + b \cdot v(Y) + c \cdot w(Y)
+ \sum_{i=1}^n a_i b_i (Y^i + Y^{-i}) - k(Y) = 0
\end{align}

where $a \cdot u(Y) + b \cdot v(Y) + c \cdot w(Y)$ is embeded into the constant term of the polynomial $t(X, Y)$.


Define $r(X, Y)$ s.t. $r(X, Y) = r(XY, 1)$.

$$\Longrightarrow r(X, Y) = \sum_{i=1}^n (a_i X^i Y^i + b_i X^{-i} Y^{-i} + c_i X^{-i-n} Y^{-i-n})$$

$$s(X, Y) = \sum_{i=1}^n (u_i(Y) X^{-i} + v_i(Y) X^i + w_i(Y) X^{i+n})$$

$$r'(X, Y) = r(X, Y) + s(X, Y)$$
$$t(X, Y) = r(X, Y) + r'(X, Y) - k(Y)$$

The coefficient of $X^0$ in $t(X, Y)$ is the left-hand side of the equation.

Sonic demonstrates that the constant term of $t(X, Y)$ is zero, thus demonstrating that our constraint system is satisfied.


\subsubsection{The basic Sonic protocol}

\begin{enumerate}[1.]
    \item Prover constructs $r(X, Y)$ using their hidden witness
    \item Prover commits to $r(X, 1)$, setting the maximum degree to n
    \item Verifier sends random challenge $y$
    \item Prover commits to $t(X, y)$. The commitment scheme ensures that $t(X, y)$ has no constant term.
    \item Verifier sends random challenge $z$
    \item Prover opens commitments to $r(z, 1), r(z, y), t(z, y)$
    \item Verifier calculates $r'(z, y)$, and checks that
	$$r(z, y) \cdot r'(z, y) - k(y) == t(z, y)$$
\end{enumerate}

Steps $3$ and $5$ can be made non-interactive by the Fiat-Shamir transformation.

\subsubsection{Polynomial Commitment Scheme}
Sonic uses an adaptation of KZG \cite{kzg-tmp}, want:

\begin{enumerate}[i.]
    \item \emph{evaluation binding}, i.e. given a commitment $F$, an adversary cannot open F to two different evaluations $v_1$ and $v_2$
    \item \emph{bounded polynomial extractable}, i.e. any algebraic adversary that opens a commitment $F$ knows an opening $f(X)$ with powers $-d \leq i \leq max, i \neq 0$.
\end{enumerate}

\vspace{0.5cm}
PC scheme (adaptation of KZG):

\begin{enumerate}[i.]
    \item Commit(info, $f(X)$) $\longrightarrow F$:
	$$F = g^{\alpha \cdot x^{d-max}} \cdot f(x)$$
    \item Open(info, $F$, $z$, $f(x)$) $\longrightarrow (f(z), W)$:
	$$w(X) = \frac{f(X) - f(z)}{X-z}$$
	$$W = g^{w(x)}$$
    \item Verify(info, $F$, $z$, $(v, W)$) $\longrightarrow 0/1$:\\
	Check:
	$$e(W, h^{\alpha \cdot x}) \cdot
	e(g^v W^{-z}, h^{\alpha})
	== e(F, h^{x^{-d+max}})$$
\end{enumerate}

\subsection{Succint signatures of correct computation}
Signature of correct computation to ensure that an element $s=s(z, y)$ for a known polynomial
$$s(X, Y) = \sum_{i, j = -d}^d s_{i, j} \cdot X^i \cdot Y^i$$

Use the structure of $s(X, Y)$ to prove its correct calculation using a \emph{permutation argument} $\longrightarrow$ \emph{grand-product argument} inspired by Bayer and Groth, and Bootle et al.

Restrict to constraint systems where $s(X, Y)$ can be expressed as the sum of $M$ polynomials. Where $j-th$ poly is of the form:
$$
\Psi_j(X, Y) =
    \sum_{i=1}^n \psi_{j, \sigma_{j, i}}
    \cdot X^i \cdot Y^{\sigma_{j, i}}
$$

where $\sigma_j$ is the fixed polynomial permutation, and $\phi_{j, i} \in \mathbb{F}$ are the coefficients.

\vspace{1cm}
\framebox{WIP}
\vspace{1cm}



\bibliography{paper-notes.bib}
\bibliographystyle{unsrt}

\end{document}
