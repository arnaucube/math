\documentclass{article}
\usepackage[utf8]{inputenc}
\usepackage{amsfonts}
\usepackage{amsthm}
\usepackage{amsmath}
\usepackage{enumerate}
\usepackage{hyperref}
\hypersetup{
    colorlinks,
    citecolor=black,
    filecolor=black,
    linkcolor=black,
    urlcolor=blue
}
\usepackage{xcolor}

% prevent warnings of underfull \hbox:
\usepackage{etoolbox}
\apptocmd{\sloppy}{\hbadness 4000\relax}{}{}

\theoremstyle{definition}
\newtheorem{definition}{Def}[section]
\newtheorem{theorem}[definition]{Thm}


\title{Paper notes}
\author{arnaucube}
\date{}

\begin{document}

\maketitle

\begin{abstract}
	Notes taken while reading papers. Usually while reading papers I take handwritten notes, this document contains some of them re-written to $LaTeX$.

	The notes are not complete, don't include all the steps neither all the proofs.
\end{abstract}

\tableofcontents

\section{SnarkPack}
Notes taken while reading SnarkPack paper \cite{cryptoeprint:2021/529}.

Groth16 proof aggregation.

\begin{enumerate}[i.]
    \item Simple verification:\\
	Proof: $\pi_i = (A_i, B_i, C_i)$\\
	Verifier checks: $e(A_i, B_i) == e(C_i, D)$\\
	Where $D$ is the $CRS$.
    \item Batch verification:
	$r \in^\$ F_q$\\
	$r^i \cdot e(A_i, B_i) == e(C_i, D)$\\
	$\Longrightarrow \prod e(A_i, B_i)^{r^i} == \prod e(C_i, D)^{r^i}$\\
	$\Longrightarrow \prod e(A_i, B_i^{r^i}) == \prod e(C_i^{r^i}, D)$
    \item Snark Aggregation verification:\\
	$z_{AB} = \prod e(A_i, B_i^{r^i})$\\
	$z_C = \prod C_i^{r^i}$\\
	Verification: $z_{AB} == e(z_C, D)$
\end{enumerate}

\section{Sonic}
Notes taken while reading Sonic paper \cite{cryptoeprint:2019/099}. Does not include all the steps, neither the proofs.

\subsection{Structured Reference String}
$\{ \{g^{x^i}\}_{i=-d}^d, \{ g^{\alpha x^i} \}_{i=-d, i \neq 0}^d, \{ h^{x^i}, h^{\alpha x^i} \}_{i=-d}^d, e(g, h^\alpha) \}$

\subsection{System of constraints}
Multiplication constraint: $a \cdot b = c$

$Q$ linear constraints:
$$
a \cdot u_q + b \cdot v_q + c \cdot w_q = k_q
$$

with $u_q, v_q, w_q \in \mathbb{F}^n$, and $k_q \in \mathbb{F}_p$.

\vspace{0.5cm}
Example: $x^2 + y^2 = z$

$$a = (x, y), \qquad b = (x, y), \qquad c = (x^2, y^2)$$
\begin{enumerate}[i.]
    \item $(x, y) \cdot (1, 0) + (x, y) \cdot (-1, 0) + (x^2, y^2) \cdot (0, 0) = 0 \longrightarrow x - x = 0$
    \item $(x, y) \cdot (0, 1) + (x, y) \cdot (0, -1) + (x^2, y^2) \cdot (0, 0) = 0 \longrightarrow y - y = 0$
    \item $(x, y) \cdot (0, 0) + (x, y) \cdot (0, 0) + (x^2, y^2) \cdot (1, 1) = z \longrightarrow x^2 + y^2 = z$
\end{enumerate}

So,
$$u_1 = (1, 0) \quad v_1=(-1, 0) \quad w_1=(0, 0) \quad k_1=0$$
$$u_2 = (0, 1) \quad v_2=(0, -1) \quad w_2=(0, 0) \quad k_2=0$$
$$u_3 = (0, 0) \quad v_3=(0, 0) \quad w_3=(1, 1) \quad k_2=z$$

\vspace{1cm}

Compress n multiplication constraints into an equation in formal indeterminate $Y$:
$$\sum_{i=1}^n (a_i b_i - c_i) \cdot Y^i = 0$$
encode into negative exponents of $Y$:
$$\sum_{i=1}^n (a_i b_i - c_i) \cdot Y^-i = 0$$

Also, compress the $Q$ linear constraints, scaling by $Y^n$ to preserve linear independence:
$$
\sum_{q=1}^Q (a \cdot u_q + b \cdot v_q + c \cdot w_q - k_q) \cdot Y^{q+n} = 0
$$

Polys:

\begin{align}
\nonumber & u_i(Y) = \sum_{q=1}^Q Y^{q+n} \cdot u_{q, i}\\
\nonumber & v_i(Y) = \sum_{q=1}^Q Y^{q+n} \cdot v_{q, i}\\
\nonumber & w_i(Y) = -Y^i - Y^{-1} + \sum_{q=1}^Q Y^{q+n} \cdot w_{q, i}\\
\nonumber & k(Y) = \sum_{q=1}^Q Y^{q+n} \cdot k_q
\end{align}

Combine the multiplicative and linear constraints to:

\begin{align}
\nonumber & a \cdot u(Y) + b \cdot v(Y) + c \cdot w(Y)
+ \sum_{i=1}^n a_i b_i (Y^i + Y^{-i}) - k(Y) = 0
\end{align}

where $a \cdot u(Y) + b \cdot v(Y) + c \cdot w(Y)$ is embeded into the constant term of the polynomial $t(X, Y)$.


Define $r(X, Y)$ s.t. $r(X, Y) = r(XY, 1)$.

$$\Longrightarrow r(X, Y) = \sum_{i=1}^n (a_i X^i Y^i + b_i X^{-i} Y^{-i} + c_i X^{-i-n} Y^{-i-n})$$

$$s(X, Y) = \sum_{i=1}^n (u_i(Y) X^{-i} + v_i(Y) X^i + w_i(Y) X^{i+n})$$

$$r'(X, Y) = r(X, Y) + s(X, Y)$$
$$t(X, Y) = r(X, Y) + r'(X, Y) - k(Y)$$

The coefficient of $X^0$ in $t(X, Y)$ is the left-hand side of the equation.

Sonic demonstrates that the constant term of $t(X, Y)$ is zero, thus demonstrating that our constraint system is satisfied.


\subsubsection{The basic Sonic protocol}

\begin{enumerate}[1.]
    \item Prover constructs $r(X, Y)$ using their hidden witness
    \item Prover commits to $r(X, 1)$, setting the maximum degree to n
    \item Verifier sends random challenge $y$
    \item Prover commits to $t(X, y)$. The commitment scheme ensures that $t(X, y)$ has no constant term.
    \item Verifier sends random challenge $z$
    \item Prover opens commitments to $r(z, 1), r(z, y), t(z, y)$
    \item Verifier calculates $r'(z, y)$, and checks that
	$$r(z, y) \cdot r'(z, y) - k(y) == t(z, y)$$
\end{enumerate}

Steps $3$ and $5$ can be made non-interactive by the Fiat-Shamir transformation.

\subsubsection{Polynomial Commitment Scheme}
Sonic uses an adaptation of KZG \cite{kzg-tmp}, want:

\begin{enumerate}[i.]
    \item \emph{evaluation binding}, i.e. given a commitment $F$, an adversary cannot open F to two different evaluations $v_1$ and $v_2$
    \item \emph{bounded polynomial extractable}, i.e. any algebraic adversary that opens a commitment $F$ knows an opening $f(X)$ with powers $-d \leq i \leq max, i \neq 0$.
\end{enumerate}

\vspace{0.5cm}
PC scheme (adaptation of KZG):

\begin{enumerate}[i.]
    \item Commit(info, $f(X)$) $\longrightarrow F$:
	$$F = g^{\alpha \cdot x^{d-max}} \cdot f(x)$$
    \item Open(info, $F$, $z$, $f(x)$) $\longrightarrow (f(z), W)$:
	$$w(X) = \frac{f(X) - f(z)}{X-z}$$
	$$W = g^{w(x)}$$
    \item Verify(info, $F$, $z$, $(v, W)$) $\longrightarrow 0/1$:\\
	Check:
	$$e(W, h^{\alpha \cdot x}) \cdot
	e(g^v W^{-z}, h^{\alpha})
	== e(F, h^{x^{-d+max}})$$
\end{enumerate}

\subsection{Succint signatures of correct computation}
Signature of correct computation to ensure that an element $s=s(z, y)$ for a known polynomial
$$s(X, Y) = \sum_{i, j = -d}^d s_{i, j} \cdot X^i \cdot Y^i$$

Use the structure of $s(X, Y)$ to prove its correct calculation using a \emph{permutation argument} $\longrightarrow$ \emph{grand-product argument} inspired by Bayer and Groth, and Bootle et al.

Restrict to constraint systems where $s(X, Y)$ can be expressed as the sum of $M$ polynomials. Where $j-th$ poly is of the form:
$$
\Psi_j(X, Y) =
    \sum_{i=1}^n \psi_{j, \sigma_{j, i}}
    \cdot X^i \cdot Y^{\sigma_{j, i}}
$$

where $\sigma_j$ is the fixed polynomial permutation, and $\phi_{j, i} \in \mathbb{F}$ are the coefficients.

\vspace{1cm}
\framebox{WIP}
\vspace{1cm}

\section{BLS signatures}
Notes taken while reading about BLS signatures \cite{bls-sig-eth2}.

\paragraph{Key generation}
$sk \in \mathbb{Z}_q$, $pk = [sk] \cdot g_1$, where $g_1 \in G_1$, and is the generator.

\paragraph{Signature}
$$\sigma = [sk] \cdot H(m)$$
where $H$ is a function that maps to a point in $G_2$. So $H(m), \sigma \in G_2$.

\paragraph{Verification}
$$e(g_1, \sigma) == e(pk, H(m))$$

Unfold:
$$e(pk, H(m)) = e([sk] \cdot g_1, H(m) = e(g_1, H(m))^{sk} = e(g_1, [sk] \cdot H(m)) = e(g_1, \sigma))$$

\paragraph{Aggregation}
Signatures aggregation:
$$\sigma_{aggr} = \sigma_1 + \sigma_2 + \ldots + \sigma_n$$
where $\sigma_{aggr} \in G_2$, and an aggregated signatures is indistinguishible from a non-aggregated signature.

\paragraph{Public keys aggregation}
$$pk_{aggr} = pk_1 + pk_2 + \ldots + pk_n$$
where $pk_{aggr} \in G_1$, and an aggregated public keys is indistinguishible from a non-aggregated public key.


\paragraph{Verification of aggregated signatures}
Identical to verification of a normal signature as long as we use the same corresponding aggregated public key:
$$e(g_1, \sigma_{aggr})==e(pk_{aggr}, H(m))$$

Unfold:
$$\fbox{e(pk_{aggr}, H(m))}= e(pk_1 + pk_2 + \ldots + pk_n, H(m)) =$$
$$=e([sk_1] \cdot g_1 + [sk_2] \cdot g_1 + \ldots + [sk_n] \cdot g_1, H(m))=$$
$$=e([sk_1 + sk_2 + \ldots + sk_n] \cdot g_1, H(m))=$$
$$=e(g_1, H(m))^{(sk_1 + sk_2 + \ldots + sk_n)}=$$
$$=e(g_1, [sk_1 + sk_2 + \ldots + sk_n] \cdot H(m))=$$
$$=e(g_1, [sk_1] \cdot H(m) + [sk_2] \cdot H(m) + \ldots + [sk_n] \cdot H(m))=$$
$$=e(g_1, \sigma_1 + \sigma_2 + \ldots + \sigma_n)= \fbox{e(g_1, \sigma_{aggr})}$$





\section{modified IPA (from Halo)}
Notes taken while reading about the modified Inner Product Argument (IPA) from the Halo paper \cite{cryptoeprint:2019/1021}.

\subsection{Notation}
\begin{description}
    \item[Scalar mul] $[a]G$, where $a$ is a scalar and $G \in \mathbb{G}$
    \item[Inner product] $<\overrightarrow{a}, \overrightarrow{b}> = a_0 b_0 + a_1 b_1 + \ldots + a_{n-1} b_{n-1}$
    \item[Multiscalar mul] $<\overrightarrow{a}, \overrightarrow{b}> = [a_0] G_0 + [a_1] G_1 + \ldots [a_{n-1}] G_{n-1}$
\end{description}


\subsection{Transparent setup}
$\overrightarrow{G} \in^r \mathbb{G}^d$, $H \in^r \mathbb{G}$

Prover wants to commit to $p(x)=a_0$
\subsection{Protocol}
Prover:
$$P=<\overrightarrow{a}, \overrightarrow{G}> + [r]H$$
$$v=<\overrightarrow{a}, \{1, x, x^2, \ldots, x^{d-1} \} >$$

where $\{1, x, x^2, \ldots, x^{d-1} \} = \overrightarrow{b}$.

We can see that computing $v$ is the equivalent to evaluating $p(x)$ at $x$ ($p(x)=v$).

We will prove:
\begin{enumerate}[i.]
    \item polynomial $p(X) = \sum a_i X^i$\\
	$p(x) = v$ (that $p(X)$ evaluates $x$ to $v$).
    \item $deg(p(X)) \leq d-1$
\end{enumerate}


Both parties know $P$, point $x$ and claimed evaluation $v$. For $U \in^r \mathbb{G}$,

$$P' = P + [v] U = <\overrightarrow{a}, G> + [r]H + [v] U$$

Now, for $k$ rounds ($d=2^k$, from $j=k$ to $j=1$):
\begin{itemize}
    \item random blinding factors: $l_j, r_j \in \mathbb{F}_p$
    \item
	$$L_j = < \overrightarrow{a}_{lo}, \overrightarrow{G}_{hi}> + [l_j] H + [< \overrightarrow{a}_{lo}, \overrightarrow{b}_{hi}>] U$$
	$$L_j = < \overrightarrow{a}_{lo}, \overrightarrow{G}_{hi}> + [l_j] H + [< \overrightarrow{a}_{lo}, \overrightarrow{b}_{hi}>] U$$
    \item Verifier sends random challenge $u_j \in \mathbb{I}$
    \item Prover computes the halved vectors for next round:
	$$\overrightarrow{a} \leftarrow \overrightarrow{a}_{hi} \cdot u_j^{-1} + \overrightarrow{a}_{lo} \cdot u_j$$
	$$\overrightarrow{b} \leftarrow \overrightarrow{b}_{lo} \cdot u_j^{-1} + \overrightarrow{b}_{hi} \cdot u_j$$
	$$\overrightarrow{G} \leftarrow \overrightarrow{G}_{lo} \cdot u_j^{-1} + \overrightarrow{G}_{hi} \cdot u_j$$
\end{itemize}

After final round, $\overrightarrow{a}, \overrightarrow{b}, \overrightarrow{G}$ are each of length 1.

Verifier can compute
$$G = \overrightarrow{G}_0 = < \overrightarrow{s}, \overrightarrow{G} >$$
and $$b = \overrightarrow{b}_0 = < \overrightarrow{s}, \overrightarrow{b} >$$
where $\overrightarrow{s}$ is the binary counting structure:

\begin{align*}
    &s = (u_1^{-1} ~ u_2^{-1} \cdots ~u_k^{-1},\\
    &~~~~~~u_1 ~~~ u_2^{-1} ~\cdots ~u_k^{-1},\\
    &~~~~~~u_1^{-1} ~~ u_2 ~~\cdots ~u_k^{-1},\\
    &~~~~~~~~~~~~~~\vdots\\
    &~~~~~~u_1 ~~~~ u_2 ~~\cdots ~u_k)
\end{align*}


And verifier checks:
$$[a]G + [r'] H + [ab] U == P' + \sum_{j=1}^k ( [u_j^2] L_j + [u_j^{-2}] R_j)$$

where the synthetic blinding factor $r'$ is $r' = r + \sum_{j=1}^k (l_j u_j^2 + r_j u_j^{-2})$.

\vspace{1cm}

Unfold:

$$
\textcolor{brown}{[a]G} + \textcolor{cyan}{[r'] H} + \textcolor{magenta}{[ab] U}
==
\textcolor{blue}{P'} + \sum_{j=1}^k ( \textcolor{violet}{[u_j^2] L_j} + \textcolor{orange}{[u_j^{-2}] R_j})
$$

\begin{align*}
&Right~side = \textcolor{blue}{P'} + \sum_{j=1}^k ( \textcolor{violet}{[u_j^2] L_j} + \textcolor{orange}{[u_j^{-2}] R_j})\\
&= \textcolor{blue}{< \overrightarrow{a}, \overrightarrow{G}> + [r] H + [v] U}\\
&+ \sum_{j=1}^k (\\
&\textcolor{violet}{[u_j^2] \cdot <\overrightarrow{a}_{lo}, \overrightarrow{G}_{hi}> + [l_j] H + [<\overrightarrow{a}_{lo}, \overrightarrow{b}_{hi}>] U}\\
&\textcolor{orange}{+ [u_j^{-2}] \cdot <\overrightarrow{a}_{hi}, \overrightarrow{G}_{lo}> + [r_j] H + [<\overrightarrow{a}_{hi}, \overrightarrow{b}_{lo}>] U}
)
\end{align*}

\begin{align*}
&Left~side = \textcolor{brown}{[a]G} + \textcolor{cyan}{[r'] H} + \textcolor{magenta}{[ab] U}\\
& = \textcolor{brown}{< \overrightarrow{a}, \overrightarrow{G} >}\\
&+ \textcolor{cyan}{[r + \sum_{j=1}^k (l_j \cdot u_j^2 + r_j u_j^{-2})] \cdot H}\\
&+ \textcolor{magenta}{< \overrightarrow{a}, \overrightarrow{b} > U}
\end{align*}


\bibliography{paper-notes.bib}
\bibliographystyle{unsrt}

\end{document}
