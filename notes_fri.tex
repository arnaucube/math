\documentclass{article}
\usepackage[utf8]{inputenc}
\usepackage{amsfonts}
\usepackage{amsthm}
\usepackage{amsmath}
\usepackage{mathtools}
\usepackage{enumerate}
\usepackage{hyperref}
\usepackage{xcolor}

% prevent warnings of underfull \hbox:
\usepackage{etoolbox}
\apptocmd{\sloppy}{\hbadness 4000\relax}{}{}

\theoremstyle{definition}
\newtheorem{definition}{Def}[section]
\newtheorem{theorem}[definition]{Thm}

% custom lemma environment to set custom numbers
\newtheorem{innerlemma}{Lemma}
\newenvironment{lemma}[1]
{\renewcommand\theinnerlemma{#1}\innerlemma}
{\endinnerlemma}


\title{Notes on FRI}
\author{arnaucube}
\date{February 2023}

\begin{document}

\maketitle

\begin{abstract}
	Notes taken from \href{https://sites.google.com/site/vincenzoiovinoit/}{Vincenzo Iovino} explainations and while reading about FRI \cite{fri}, \cite{cryptoeprint:2022/1216}.

	Usually while reading papers I take handwritten notes, this document contains some of them re-written to $LaTeX$.

	The notes are not complete, don't include all the steps neither all the proofs.
\end{abstract}

\tableofcontents

\section{Preliminaries}
\subsection{Low degree testing}
V wants to ensure that $deg(f(x)) \leq d$.

We are in the IOP setting, V asks on a point, P sends back the opening at that point.

TODO

\subsubsection{General degree d test}

Query at points $\{ x_i \}_0^{d+1},~z$ (with rand $z \overset{R}{\in} \mathbb{F}$).
Interpolate $p(x)$ at $\{f(x_i)\}_0^{d+1}$ to reconstruct the unique polynomial $p$ of degree $d$ such that $p(x_i)=f(x_i)~\forall i=1, \ldots, d+1$.

V checks $p(z)=f(z)$, if the check passes, then V is convinced with high probability.

This needs $d+2$ queries, is linear, $\mathcal{O}(n)$. With FRI we will have the test in $\mathcal{O}(\log{}d)$.

\section{FRI protocol}
Allows to test if a function $f$ is a poly of degree $\leq d$ in $\mathcal{O}(\log{}d)$.

Note: "P \emph{sends} $f(x)$ to V", "\emph{sends}", in the ideal IOP model means that all the table of $f(x)$ is sent, in practice is sent a commitment to $f(x)$.

\subsection{Intuition}
V wants to check that two functions $g,~h$ are both polynomials of degree $\leq d$.

Consider the following protocol:

\begin{enumerate}
	\item V sends $\alpha \in \mathbb{F}$ to P. P sends $f(x) = g(x) + \alpha h(x)$ to V.
	\item P sends $f(x)=g(x) + \alpha h(x)$ to V.
	\item V queries $f(r), ~g(r), ~h(r)$ for rand $r \in \mathbb{F}$.
	\item V checks $f(r)=g(r) + \alpha h(r)$. (Schwartz-Zippel lema).
		If holds, V can be certain that $f(x)=g(x)+ \alpha h(x)$.
	\item P proves that $deg(f) \leq d$. 
	\item If V is convinced that $deg(f) \leq d$, V belives that both $g, h$ have $deg \leq d$.
\end{enumerate}

%/// TODO tabulate this next lines
With high probablility, $\alpha$ will not cancel the coeffs with $deg \geq d+1$. % TODO check which is the name of this theorem or why this is true

Let $g(x)=a \cdot x^{d+1}, ~~ h(x)=b \cdot x^{d+1}$, and set $f(x) = g(x) + \alpha h(x)$.
Imagine that P can chose $\alpha$ such that $a x^{d+1} + \alpha \cdot b x^{d+1} = 0$, then, in $f(x)$ the coefficients of degree $d+1$ would cancel.
%///

\quad

Here, P proves $g,~h$ both have $deg \leq d$, but instead of doing $2 \cdot (d+2)$ queries ($d+2$ for $g$, and $d+2$ for $h$), it is done in $d+2$ queries (for $f$).
So we halved the number of queries.


\subsection{FRI}
Both P and V have oracle access to function $f$.

V wants to test if $f$ is polynomial with $deg(f) \leq d$.

Let $f_0(x)=f(x)$.

Each polynomial $f(x)$ of degree that is a power of $2$, can be written as
$$f(x) = f^L(x^2) + x f^R(x^2)$$
for some polynomials $f^L,~f^R$ of degree $\frac{deg(f)}{2}$, each one containing the even and odd degree coefficients as follows:

% $f^L(x)$ is built from the even degree coefficients divided by $x$, and $f^R(x)$ from the odd degree coefficients divided by $x$.

$$f^L(x)= \sum_0^{\frac{d+1}{2}-1} c_{2i} x^i ,~~ f^R(x)= \sum_0^{\frac{d+1}{2}-1} c_{2i+1} x^i$$

eg. for $f(x)=x^4+x^3+x^2+x+1$,
\begin{align*}
	\begin{rcases}
	f^L(x)=x^2+x+1\\
	f^R(x)=x+1
	\end{rcases}
	~f(x) = f^L(x^2) &+ x \cdot f^R(x^2)\\
	= (x^2)^2 + (x^2) + 1 &+ x \cdot ((x^2) + 1)\\
	= x^4 + x^2 + 1 &+ x^3 + x
\end{align*}

\begin{enumerate}
	\item V sends to P some $\alpha_0 \in \mathbb{F}$.
		Let
		\begin{equation}\tag{$A_0$}
			f_0(x) = f_0^L(x^2) + x f_0^R(x^2)
		\end{equation}
	\item P sends
		\begin{equation}\tag{$B_0$}
			f_1(x) = f_0^L(x) + \alpha_0 f_0^R(x)
		\end{equation}
		to V.

		(remember that "sends" in IOP model is that P commits to it)
	\item V sends to P some $\alpha_1 \in \mathbb{F}$.
		Let
		\begin{equation}\tag{$A_1$}
			f_1(x) = f_1^L(x^2) + x f_1^R(x^2)
		\end{equation}
	\item P sends
		\begin{equation}\tag{$B_1$}
			f_2(x) = f_1^L(x) + \alpha_1 f_1^R(x)
		\end{equation}
		to V.
	\item Keep repeating the process, eg. let
		\begin{equation}\tag{$A_2$}
			f_2(x) = f_2^L(x^2) + x f_2^R(x^2)
		\end{equation}
		until $f_i^L,~ f_i^R$ are constant (degree 0 polynomials).
	\item Once $f_i^L,~ f_i^R$ are constant, P sends them to V.
\end{enumerate}

Notice that at each step, $deg(f_i)$ halves.

\paragraph{Query phase}

\begin{enumerate}
	\item V sends rand $z \in \mathbb{F}$ to P
	\item P sends $\{ f_i(z^{2^i}), f_i(- z^{2^i}) \}$ to V.\\
		{\scriptsize eg. $f_0(z),~ f_0(-z),~ f_1(z^2),~ f_1(-z^2),~ f_2(z^4),~ f_2(-z^4),~ f_3(z^8),~ f_3(-z^8),~ \ldots$}
	\item V checks $f_i(a)=f_i^L(a^2) + a f_i^R(a^2)$ for $a=\{z, -z\}$
		$$
		\begin{pmatrix}
			1 & z\\
			1 & -z
		\end{pmatrix}
		\begin{pmatrix}
			f_i^L(z^2)\\
			f_i^R(z^2)
		\end{pmatrix}
		=
		\begin{pmatrix}
			f_i(z)\\
			f_i(-z)
		\end{pmatrix}
		$$
\end{enumerate}

The number of queries needed is $2 \cdot log(d)$.

\section{FRI as polynomial commitment}
\emph{[WIP. Unfinished document]}


\bibliography{paper-notes.bib}
\bibliographystyle{unsrt}

\end{document}
