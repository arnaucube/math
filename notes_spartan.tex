\documentclass{article}
\usepackage[utf8]{inputenc}
\usepackage{amsfonts}
\usepackage{amsthm}
\usepackage{amsmath}
\usepackage{mathtools}
\usepackage{enumerate}
\usepackage{hyperref}
\usepackage{xcolor}
\usepackage{pgf-umlsd} % diagrams
\usepackage{centernot}


% prevent warnings of underfull \hbox:
\usepackage{etoolbox}
\apptocmd{\sloppy}{\hbadness 4000\relax}{}{}

\theoremstyle{definition}
\newtheorem{definition}{Def}[section]
\newtheorem{theorem}[definition]{Thm}

% custom lemma environment to set custom numbers
\newtheorem{innerlemma}{Lemma}
\newenvironment{lemma}[1]
{\renewcommand\theinnerlemma{#1}\innerlemma}
{\endinnerlemma}


\title{Notes on Spartan}
\author{arnaucube}
\date{April 2023}

\begin{document}

\maketitle

\begin{abstract}
	Notes taken while reading about Spartan \cite{cryptoeprint:2019/550}.

	Usually while reading papers I take handwritten notes, this document contains some of them re-written to $LaTeX$.

	The notes are not complete, don't include all the steps neither all the proofs.
\end{abstract}

\tableofcontents

\section{R1CS into Sum-Check protocol}
\begin{definition}{R1CS}
	$\exists w \in \mathbb{F}^{m - |io| - 1}$ such that $(A \cdot z) \circ (B \cdot z) = (C \cdot z)$, where $z=(io, 1, w)$.
\end{definition}


\textbf{Thm 4.1} $\forall$ R1CS instance $x = (\mathbb{F}, A, B, C, io, m, n)$, $\exists$ a degree-3 log m-variate polynomial $G$ such that $\sum_{x \in \{0,1\}^{log m}} G(x) = 0$ iff $\exists$ a witness $w$ such that $Sat_{R1CS}(x, w)=1$.
% \begin{theorem}{4.1} // TODO use theorem gadget
%	$\forall$
% \begin{end}
\vspace{0.5cm}

% For a RCS instance $x$, let $s = \lceil \log m \rceil$.

We can view matrices $A, B, C \in \mathbb{F}^{m \times m}$ as functions $\{0,1\}^s \times \{0,1\}^s \rightarrow \mathbb{F}$ ($s= \lceil \log m \rceil$).
For a given witness $w$ to $x$, let $z=(io, 1, w)$.
View $z$ as a function $\{0,1\}^s \rightarrow \mathbb{F}$, so any entry in $z$ can be accessed with a $s$-bit identifier.

\begin{small}
$$
F_{io}(x)=\left( \sum_{y \in \{0,1\}^s} A(x, y) \cdot Z(y) \right) \cdot \left( \sum_{y \in \{0,1\}^s} B(x, y) \cdot Z(y) \right) - \sum_{y \in \{0,1\}^s} C(x, y) \cdot Z(y)
$$
\end{small}

\begin{lemma}{4.1}
	$\forall x \in \{0,1\}^s,~ F_{io}(x)=0$ iff $Sat_{R1CS}(x,w)=1$.
\end{lemma}

$F_{io}(\cdot)$ is a function, not a polynomial, so it can not be used in the Sum-check protocol.

$F_{io}(x)$ function is converted to a polynomial by using its polynomial extension $\widetilde{F}_{io}(x): \mathbb{F}^s \rightarrow \mathbb{F}$,
\begin{small}
$$
\widetilde{F}_{io}(x)=\left( \sum_{y \in \{0,1\}^s} \widetilde{A}(x, y) \cdot \widetilde{Z}(y) \right) \cdot \left( \sum_{y \in \{0,1\}^s} \widetilde{B}(x, y) \cdot \widetilde{Z}(y) \right) - \sum_{y \in \{0,1\}^s} \widetilde{C}(x, y) \cdot \widetilde{Z}(y)
$$
\end{small}

\begin{lemma}{4.2}
	$\forall x \in \{0,1\}^s,~ \widetilde{F}_{io}(x)=0$ iff $Sat_{R1CS}(x, w)=1$.
\end{lemma}

(proof: $\forall x \in \{0,1\}^s,~ \widetilde{F}_{io}(x)=F_{io}(x)$, so, result follows from Lemma 4.1.) % TODO link to lemma

\vspace{0.5cm}

So, for this, V will need to check that $\widetilde{F}_{io}$ vanishes over the boolean hypercube ($\widetilde{F}_{io}(x)=0 ~\forall x \in \{0,1\}^s$).

Recall that $\widetilde{F}_{io}(\cdot)$ is a low-degree multivariate polynomial over $\mathbb{F}$ in $s$ variables.
Thus, checking that $\widetilde{F}_{io}$ vanishes over the boolean hypercube is equivalent to checking that $\widetilde{F}_io=0$.

Thus, V can check $\sum_{x \in \{0,1\}^s} \widetilde{F}_{io}(x)=0$ using the Sum-check protocol (through SZ lemma, V can check if for a random value it equals to 0, and be convinced that applies to all the points whp.).

But: as $\widetilde{F}_{io}(x)$ is not multilinear, so $\sum_{x\in \{0,1\}^s} \widetilde{F}_{io}(x)=0 \centernot\Longleftrightarrow F_{io}(x)=0 ~\forall x \in \{0,1\}^s$.
Bcs: the $2^s$ terms in the sum might cancel each other even when the individual terms are not zero.

Solution: combine $\widetilde{F}_{io}(x)$ with $\widetilde{eq}(t, x)$ to get $Q_{io}(t, x)$ which will be the unique multilinear polynomial, and then check that it is a zero-polynomial

$$Q_{io}(t)= \sum_{x \in \{0,1\}^s} \widetilde{F}_{io}(x) \cdot \widetilde{eq}(t, x)$$

where $\widetilde{eq}(t, x) = \prod_{i=1}^s (t_i \cdot x_i + (1- t_i) \cdot (1- x_i))$, which is the MLE of $eq(x,e)= \{ 1 ~\text{if}~ x=e,~ 0 ~\text{otherwise} \}$.

Basically $Q_{io}(\cdot)$ is a multivariate (the unique multilinear) polynomial such that
$$Q_{io}(t) = \widetilde{F}_{io}(t) ~\forall t \in \{0,1\}^s$$
thus, $Q_{io}(\cdot)$ is a zero-polynomial iff $\widetilde{F}_{io}(x)=0 ~\forall x\in \{0,1\}^s$.
$\Longleftrightarrow$ iff $\widetilde{F}_{io}(\cdot)$ encodes a witness $w$ such that $Sat_{R1CS}(x, w)=1$.

$\widetilde{F}_{io}(x)$ has degree 2 in each variable, and $\widetilde{eq}(t, x)$ has degree 1 in each variable, so $Q_{io}(t)$ has degree 3 in each variable.

To check that $Q_{io}(\cdot)$ is a zero-polynomial: check $Q_{io}(\tau)=0,~ \tau \in^R \mathbb{F}^s$ (Schwartz-Zippel-DeMillo–Lipton lemma) through the sum-check protocol.

This would mean that the R1CS instance is satisfied.


\paragraph{Recap}
\begin{itemize}
	\item[] We have that $Sat_{R1CS}(x,w)=1$ iff $F_{io}(x)=0$.
	\item[] To be able to use sum-check, we use its polynomial extension $\widetilde{F}_{io}(x)$, using sum-check to prove that $\widetilde{F}_{io}(x) =0 ~\forall x \in \{0, 1\}^s$, which means that $Sat_{R1CS}(x,~w)=1$.
	\item[] To prevent potential canceling terms, we combine $\widetilde{F}_{io}(x)$ with $\widetilde{eq}(t, x)$, obtaining $G_{io, \tau}(x)= \widetilde{F}_{io}(x) \cdot \widetilde{eq}(t, x)$.
	\item[] Thus $Q_{io}(t)= \sum_{x \in \{0,1\}^s} \widetilde{F}_{io}(x) \cdot \widetilde{eq}(t, x)$, and then we prove that $Q_{io}(\tau)=0$, for $\tau \in^R \mathbb{F}^s$.
\end{itemize}

\section{NIZKs with succinct proofs for R1CS}

From Thm 4.1: to check R1CS instance $(\mathbb{F}, A, B, C, io, m, n)$ V can check if
$\sum_{x \in \{0,1\}^s} G_{io, \tau} (x) = 0$, which through sum-check protocol can be reduced to $e_x = G_{io, \tau} (r_x)$, where $r_x \in \mathbb{F}^s$.

Recall: $G_{io, \tau}(x) = \widetilde{F}_{io}(x) \cdot \widetilde{eq}(\tau, x)$.

Evaluating $\widetilde{eq}(\tau, r_x)$ takes $O(log~m)$, but to evaluate $\widetilde{F}_{io}(r_x)$, V needs to evaluate
$$\widetilde{A}(r_x, y), \widetilde{B}(r_x, y), \widetilde{C}(r_x, y), \widetilde{Z}(y),~ \forall y \in \{0,1\}^s$$

which requires 3 sum-check instances (\begin{scriptsize}
$\left( \sum_{y \in \{0,1\}^s} \widetilde{A}(x, y) \cdot \widetilde{Z}(y) \right)$,\\ $\left( \sum_{y \in \{0,1\}^s} \widetilde{B}(x, y) \cdot \widetilde{Z}(y) \right)$, $\left( \sum_{y \in \{0,1\}^s} \widetilde{C}(x, y) \cdot \widetilde{Z}(y) \right)$
\end{scriptsize}), one for each summation in\\ $\widetilde{F}_{io}(x)$.

But note that evaluations of $\widetilde{Z}(y) ~\forall y \in \{0,1\}^s$ are already known as $(io, 1, w)$.

Solution: combination of 3 protocols:
\begin{itemize}
	\item Sum-check protocol
	\item randomized mini protocol
	\item polynomial commitment scheme
\end{itemize}
Basically to do a random linear combination of the 3 summations to end up doing just a single sum-check.

Observation: let $\widetilde{F}_{io}(r_x) = \overline{A}(r_x) \cdot \overline{B}(r_x) - \overline{C}(r_x)$, where
$$\overline{A}(r_x) = \sum_{y \in \{0,1\}} \widetilde{A}(r_x, y) \cdot \widetilde{Z}(y),~~\overline{B}(r_x) = \sum_{y \in \{0,1\}} \widetilde{B}(r_x, y) \cdot \widetilde{Z}(y)$$
$$\overline{C}(r_x) = \sum_{y \in \{0,1\}} \widetilde{C}(r_x, y) \cdot \widetilde{Z}(y)$$

Prover makes 3 separate claims: $\overline{A}(r_x)=v_A,~ \overline{B}(r_x)=v_B,~ \overline{C}(r_x)=v_C$,
then V evaluates:
$$G_{io, \tau}(r_x) = (v_A \cdot v_B - v_C) \cdot \widetilde{eq}(r_x, \tau)$$


\begin{footnotesize}
	which equals to
	$$=\left(\overline{A}(r_x) \cdot \overline{B}(r_x) - \overline{C}(r_x)\right) \cdot \widetilde{eq}(r_x, \tau)=$$
	$$\left(\left(\sum_{y \in \{0,1\}} \widetilde{A}(r_x, y) \cdot \widetilde{Z}(y)\right) \cdot \left(\sum_{y \in \{0,1\}} \widetilde{B}(r_x, y) \cdot \widetilde{Z}(y)\right) - \sum_{y \in \{0,1\}} \widetilde{C}(r_x, y) \cdot \widetilde{Z}(y)\right) \cdot \widetilde{eq}(r_x, \tau)$$
\end{footnotesize}

\vspace{0.5cm}

This would be 3 sum-check protocol instances (3 claims: $\overline{A}(r_x)=v_A$, $\overline{B}(r_x)=v_B$, $\overline{C}(r_x)=v_C$).

Instead, combine 3 claims into a single claim:

\begin{itemize}
	\item V samples $r_A, r_B, r_C \in^R \mathbb{F}$, and computes $c= r_A v_A + r_B v_B + r_C v_C$.
	\item V, P use sum-check protocol to check:
$$r_A \cdot \overline{A}(r_x) + r_B \cdot \overline{B}(r_x) + r_C \cdot \overline{C}(r_x) == c$$


% Let $L(r_x) = r_A \cdot \overline{A}(r_x) +r_B \cdot \overline{B}(r_x) +r_C \cdot \overline{C}(r_x)$,
Let
\begin{small}
\begin{align*}
	&L(r_x) = r_A \cdot \overline{A}(r_x) +r_B \cdot \overline{B}(r_x) +r_C \cdot \overline{C}(r_x)\\
	&= \sum_{y \in \{0,1\}^s}
	       \left( r_A \cdot \widetilde{A}(r_x, y) \cdot \widetilde{Z}(y)
	       + r_B \cdot \widetilde{B}(r_x, y) \cdot \widetilde{Z}(y)
	       + r_C \cdot \widetilde{C}(r_x, y) \cdot \widetilde{Z}(y) \right)\\
	       &= \sum_{y \in \{0,1\}^s} M_{r_x}(y)
\end{align*}
\end{small}

$M_{r_x}(y)$ is a s-variate polynomial with deg $\leq 2$ in each variable ($\Longleftrightarrow \mu = s,~ l=2,~ T=c$).

\end{itemize}


\begin{align*}
M_{r_x}(r_y) &=
r_A \cdot \widetilde{A}(r_x, r_y) \cdot \widetilde{Z}(r_y)
+ r_B \cdot \widetilde{B}(r_x, r_y) \cdot \widetilde{Z}(r_y)
+ r_C \cdot \widetilde{C}(r_x, r_y) \cdot \widetilde{Z}(r_y)\\
	     &=
	     (r_A \cdot \widetilde{A}(r_x, r_y)
+ r_B \cdot \widetilde{B}(r_x, r_y)
+ r_C \cdot \widetilde{C}(r_x, r_y)) \cdot \widetilde{Z}(r_y)\\
\end{align*}

only one term in $M_{r_x}(r_y)$ depends on prover's witness: $\widetilde{Z}(r_y)$, the other terms can be computed locally by V in $O(n)$ time (Section 6 of the paper for sub-linear in $n$).

Instead of evaluating $\widetilde{Z}(r_y)$ in $O(|w|)$ communications, P sends a commitment to $\widetilde{w}(\cdot)$ (= MLE of the witness $w$) to V before the first instance of the sum-check protocol.


\paragraph{Recap}
\begin{itemize}
	\item[] To check the R1CS instance, V can check $\sum_{x \in \{0,1\}^s} G_{io, \tau} (x) = 0$, which through the sum-check is reduced to $e_x = G_{io, \tau} (r_x)$, for $r_x \in \mathbb{F}^s$.
	\item[] Evaluating $G_{io, \tau}(x)$ ($G_{io, \tau}(x) = \widetilde{F}_{io}(x) \cdot \widetilde{eq}(\tau, x)$) is not cheap. Evaluating $\widetilde{eq}(\tau, r_x)$ takes $O(log~m)$, but to evaluate $\widetilde{F}_{io}(r_x)$, V needs to evaluate $\widetilde{A}, \widetilde{B}, \widetilde{C}, \widetilde{Z},~ \forall y \in \{0,1\}^s$
	% \item[] Solution: combine 3 protocols: sum-check protocol, randomized mini protocol, polynomial commitment scheme.
	\item[] P makes 3 separate claims: $\overline{A}(r_x)=v_A,~ \overline{B}(r_x)=v_B,~ \overline{C}(r_x)=v_C$, so V can evaluate $G_{io, \tau}(r_x) = (v_A \cdot v_B - v_C) \cdot \widetilde{eq}(r_x, \tau)$
	\item[] The previous claims are combined into a single claim (random linear combination) to use only a single sum-check protocol:
		\begin{itemize}
			\item[] P: $c= r_A v_A + r_B v_B + r_C v_C$, for $r_A, r_B, r_C \in^R \mathbb{F}$
			\item[] V, P: sum-check $r_A \cdot \overline{A}(r_x) + r_B \cdot \overline{B}(r_x) + r_C \cdot \overline{C}(r_x) == c$
		\end{itemize}
	\item[] $c=L(r_x)=\sum_{y \in \{0,1\}^s} M_{r_x}(y)$, where $M_{r_x}(y)$ is a s-variate polynomial with deg $\leq 2$ in each variable ($\Longleftrightarrow \mu = s,~ l=2,~ T=c$). Only $\widetilde{Z}(r_y)$ depends on P's witness, the other terms can be computed locally by V.
	\item[] Instead of evaluating $\widetilde{Z}(r_y)$ in $O(|w|)$ communications, P uses a commitment to $\widetilde{w}(\cdot)$ (= MLE of the witness $w$).
\end{itemize}

\subsection{Full protocol}
\begin{footnotesize}
	(Recall: Sum-Check params: $\mu$: n vars, n rounds, $l$: degree in each variable upper bound, $T$: claimed result.)
\end{footnotesize}

\begin{itemize}
	\item $pp \leftarrow Setup(1^{\lambda})$: invoke $pp \leftarrow PC.Setup(1^{\lambda}, log m)$; output $pp$
	\item $b \leftarrow <P(w), V(r)>(\mathbb{F}, A,B,C, io, m, n)$:
	\begin{enumerate}
		\item P: $(C, S) \leftarrow PC.Commit(pp, \widetilde{w})$ and send $C$ to V
		\item V: send $\tau \in^R \mathbb{F}^{log~m}$ to P
		\item let $T_1=0,~ \mu_1=log~m,~ l_1=3$
		\item V: set $r_x \in^R \mathbb{F}^{\mu_1}$
		\item Sum-check 1. $e_x \leftarrow <P_{SC}(G_{io,\tau}), V_{SC}(r_x)>(\mu_1, l_1, T_1)$
		\item P: compute $v_A=\overline{A}(r_x),~ v_B=\overline{B}(r_x),~ v_C=\overline{C}(r_x)$, send $(v_A, v_B, v_C)$ to V
		\item V: abort with $b=0$ if $e_x \neq (v_A \cdot v_B - v_C)\cdot \widetilde{eq}(r_x, \tau)$
		\item V: send $r_A, r_B, r_C \in^R \mathbb{F}$ to P
		\item let $T_2 = r_A \cdot v_A + r_B \cdot v_B + r_C \cdot v_C,~ \mu_2=log~m,~ l_2=2$
		\item V: set $r_y \in^R \mathbb{F}^{\mu_2}$
		\item Sum-check 2. $e_y \leftarrow <P_{SC}(M_{r_x}), V_{SC}(r_y)>(\mu_2, l_2, T_2)$
		\item P: $v \leftarrow \widetilde{w}(r_y[1..])$, send $v$ to V
		\item $b_e \leftarrow <P_{PC.Eval}(\widetilde{w}, S), V_{PC.Eval}(r)>(pp,  C, r_y, v, \mu_2)$
		\item V: abort with $b=0$ if $b_e==0$
		\item V: $v_z \leftarrow (1 - r_y[0]) \cdot \widetilde{w}(r_y [1..]) + r_y[0] \widetilde{(io, 1)} (r_y[1..])$
		\item V: $v_1 \leftarrow \widetilde{A}(r_x, r_y),~ v_2 \leftarrow \widetilde{B}(r_x, r_y),~ v_3 \leftarrow \widetilde{C}(r_x, r_y)$
		\item V: abort with $b=0$ if $e_y \neq (r_A v_1 + r_B v_2 + r_C v_3) \cdot v_z$
		\item V: output $b=1$
	\end{enumerate}
\end{itemize}

Section 6 of the paper, describes how in step 16, instead of evaluating $\widetilde{A},~\widetilde{B},~\widetilde{C}$ at $r_x,~r_y$ with $O(n)$ costs, P commits to $\widetilde{A},~\widetilde{B},~\widetilde{C}$ and later provides proofs of openings.

In a practical implementation those commits to $\widetilde{A},~\widetilde{B},~\widetilde{C}$ could be done in a preprocessing step.

\vspace{1cm}
\framebox{WIP: covered until sec.6}



\bibliography{paper-notes.bib}
\bibliographystyle{unsrt}

\end{document}
