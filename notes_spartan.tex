\documentclass{article}
\usepackage[utf8]{inputenc}
\usepackage{amsfonts}
\usepackage{amsthm}
\usepackage{amsmath}
\usepackage{mathtools}
\usepackage{enumerate}
\usepackage{hyperref}
\usepackage{xcolor}
\usepackage{pgf-umlsd} % diagrams
\usepackage{centernot}


% prevent warnings of underfull \hbox:
\usepackage{etoolbox}
\apptocmd{\sloppy}{\hbadness 4000\relax}{}{}

\theoremstyle{definition}
\newtheorem{definition}{Def}[section]
\newtheorem{theorem}[definition]{Thm}

% custom lemma environment to set custom numbers
\newtheorem{innerlemma}{Lemma}
\newenvironment{lemma}[1]
{\renewcommand\theinnerlemma{#1}\innerlemma}
{\endinnerlemma}


\title{Notes on Spartan}
\author{arnaucube}
\date{April 2023}

\begin{document}

\maketitle

\begin{abstract}
	Notes taken while reading about Spartan \cite{cryptoeprint:2019/550}.

	Usually while reading papers I take handwritten notes, this document contains some of them re-written to $LaTeX$.

	The notes are not complete, don't include all the steps neither all the proofs.
\end{abstract}

\tableofcontents

\section{Encoding R1CS instances as low-degree polynomials}
\begin{definition}{R1CS}
	$\exists w \in \mathbb{F}^{m - |io| - 1}$ such that $(A \cdot z) \circ (B \cdot z) = (C \cdot z)$, where $z=(io, 1, w)$.
\end{definition}


\textbf{Thm 4.1} $\forall$ R1CS instance $x = (\mathbb{F}, A, B, C, io, m, n)$, $\exists$ a degree-3 log m-variate polynomial $G$ such that $\sum_{x \in \{0,1\}^{log m}} G(x) = 0$.
% \begin{theorem}{4.1} // TODO use theorem gadget
%	$\forall$
% \begin{end}
\vspace{0.5cm}

For a RCS instance $x$, let $s = \lceil log m \rceil$.

We can view matrices $A, B, C \in \mathbb{F}^{m \times m}$ as functions $\{0,1\}^s \times \{0,1\}^s \rightarrow \mathbb{F}$.
For a given witness $w$ to $x$, let $z=(io, 1, w)$.
View $z$ as a function $\{0,1\}^s \rightarrow \mathbb{F}$, so any entry in $z$ can be accessed with a $s$-bit identifier.

$$
F_{io}(x)=
$$
$$
\left( \sum_{y \in \{0,1\}^s} A(x, y) \cdot Z(y) \right) \cdot \left( \sum_{y \in \{0,1\}^s} B(x, y) \cdot Z(y) \right) - \left( \sum_{y \in \{0,1\}^s} C(x, y) \cdot Z(y) \right)
$$

\begin{lemma}{4.1}
	$\forall x \in \{0,1\}^s,~ F_{io}(x)=0$ iff $Sat_{R1CS}(x,w)=1$.
\end{lemma}

$F_{io}(\cdot)$ is a function, not a polynomial, so it can not be used in the Sum-check protocol.

consider its polynomial extension $\widetilde{F}_{io}(x): \mathbb{F}^s \rightarrow \mathbb{F}$,
$$\widetilde{F}_{io}(x)=$$
$$
\left( \sum_{y \in \{0,1\}^s} \widetilde{A}(x, y) \cdot \widetilde{Z}(y) \right) \cdot \left( \sum_{y \in \{0,1\}^s} \widetilde{B}(x, y) \cdot \widetilde{Z}(y) \right) - \left( \sum_{y \in \{0,1\}^s} \widetilde{C}(x, y) \cdot \widetilde{Z}(y) \right)
$$

\begin{lemma}{4.2}
	$\forall x \in \{0,1\}^s,~ \widetilde{F}_{io}(x)=0$ iff $Sat_{R1CS}(x, w)=1$.
\end{lemma}

(proof: $\forall x \in \{0,1\}^s,~ \widetilde{F}_{io}(x)=F_{io}(x)$, so, result follows from Lemma 4.1.) % TODO link to lemma

\vspace{0.5cm}

$\widetilde{F}_{io}(\cdot)$: low-degree multivariate polynomial over $\mathbb{F}$ in $s$ variables.
Verifier can check if $\sum_{x \in \{0,1\}^s} \widetilde{F}_{io}(x)=0$ using the Sum-check protocol.

But: $\sum_{x\in \{0,1\}^s} \widetilde{F}_{io}(x)=0 \centernot\Longleftrightarrow F_{io}(x)=0 \forall x \in \{0,1\}^s$.
Bcs: the $2^s$ terms in the sum might cancel each other even when the individual terms are not zero.
Solution: consider
$$Q_{io}(t)= \sum_{x \in \{0,1\}^s} \widetilde{F}_{io}(x) \cdot \widetilde{eq}(t, x)$$
where $\widetilde{eq}(t, x) = \prod_{i=1}^s (t_i \cdot x_i + (1- t_i) \cdot (1- x_i))$.

Basically $Q_{io}(\cdot)$ is a multivariate polynomial such that
$$Q_{io}(t) = \widetilde{F}_{io}(t) ~\forall t \in \{0,1\}^s$$
thus, $Q_{io}(\cdot)$ is a zero-polynomial iff $\widetilde{F}_{io}(x)=0 ~\forall x\in \{0,1\}^s$.
$\Longleftrightarrow$ iff $\widetilde{F}_{io}(\cdot)$ encodes a witness $w$ such that $Sat_{R1CS}(x, w)=1$.

To check that $Q_{io}(\cdot)$ is a zero-polynomial: check $Q_{io}(\tau)=0,~ \tau \in^R \mathbb{F}^s$ (Schwartz-Zippel-DeMillo–Lipton lemma).

\section{NIZKs with succint proofs for R1CS}

From Thm 4.1: to check R1CS instance $(\mathbb{F}, A, B, C, io, m, n)$ V can check if
$$\sum_{x \in \{0,1\}^s} G_{io, \tau} (r_x)$$
where $r_x \in \mathbb{F}^s$.

Recall: $G_{io, \tau}(x) = \widetilde{F}_{io}(x) \cdot \widetilde{eq}(\tau, x)$.

To evaluate $\widetilde{F}_{io}(r_x)$, V needs to evaluate
$$\forall y \in \{0,1\}^s: \widetilde{A}(r_x, y), \widetilde{B}(r_x, y), \widetilde{C}(r_x, y), \widetilde{Z}(y)$$
evaluations of $\widetilde{Z}(y) ~\forall y \in \{0,1\}^s ~\Longleftrightarrow (io, 1, w)$.

Solution: combination of 3 protocols:
\begin{itemize}
	\item Sum-check protocol
	\item randomized mini protocol
	\item polynomial commitment scheme
\end{itemize}

Observation: let $\widetilde{F}_{io}(r_x) = \bar{A}(r_x) \cdot \bar{B}(r_x) - \bar{C}(r_x)$, where
$$\bar{A}(r_x) = \sum_{y \in \{0,1\}} \widetilde{A}(r_x, y) \cdot \widetilde{Z}(y)$$
$$\bar{B}(r_x) = \sum_{y \in \{0,1\}} \widetilde{B}(r_x, y) \cdot \widetilde{Z}(y)$$
$$\bar{C}(r_x) = \sum_{y \in \{0,1\}} \widetilde{C}(r_x, y) \cdot \widetilde{Z}(y)$$

Prover makes 3 separate claims: $\bar{A}(r_x)=v_A,~ \bar{B}(r_x)=v_B,~ \bar{C}(r_x)=v_C$,
then V evaluates:
$$G_{io, \tau}(r_x) = (v_A \cdot v_B - v_C) \cdot \widetilde{eq}(r_x, \tau)$$
which could be 3 sum-check protocol instances. Instead: combine 3 claims into a single claim:

V samples $r_A, r_B, r_C \in^R \mathbb{F}$, and computes $c= r_A v_A + r_B v_B + r_C v_C$.
V, P use sum-check protocol to check:
$$r_A \cdot \bar{A}(r_x) + r_B \cdot \bar{B}(r_x) + r_C \cdot \bar{C}(r_x) == c$$


Let $L(r_x) = r_A \cdot \bar{A}(r_x) +r_B \cdot \bar{B}(r_x) +r_C \cdot \bar{C}(r_x)$,

\begin{align*}
	L(r_x) &= \sum_{y \in \{0,1\}^s}
	       r_A \cdot \widetilde{A}(r_x, y) \cdot \widetilde{Z}(y)
+ r_B \cdot \widetilde{B}(r_x, y) \cdot \widetilde{Z}(y)
+ r_C \cdot \widetilde{C}(r_x, y) \cdot \widetilde{Z}(y)\\
	       &= \sum_{y \in \{0,1\}^s} M_{r_x}(y)
\end{align*}

$M_{r_x}(y)$ is a s-variate polynomial with deg $\leq 2$ in each variable ($\Longleftrightarrow \mu = s,~ l=2,~ T=c$).


\begin{align*}
M_{r_x}(r_y) &=
r_A \cdot \widetilde{A}(r_x, r_y) \cdot \widetilde{Z}(r_y)
+ r_B \cdot \widetilde{B}(r_x, r_y) \cdot \widetilde{Z}(r_y)
+ r_C \cdot \widetilde{C}(r_x, r_y) \cdot \widetilde{Z}(r_y)\\
	     &=
	     (r_A \cdot \widetilde{A}(r_x, r_y)
+ r_B \cdot \widetilde{B}(r_x, r_y)
+ r_C \cdot \widetilde{C}(r_x, r_y)) \cdot \widetilde{Z}(r_y)\\
\end{align*}

only one term in $M_{r_x}(r_y)$ depends on prover's witness: $\widetilde{Z}(r_y)$

P sends a commitment to $\widetilde{w}(\cdot)$ (= MLE of the witness $w$) to V before the first instance of the sum-check protocol.


\subsection{Full protocol}

\begin{itemize}
	\item $pp \leftarrow Setup(1^{\lambda})$: invoke $pp \leftarrow PC.Setup(1^{\lambda}, log m)$; output $pp$
	\item $b \leftarrow <P(w), V(r)>(\mathbb{F}, A,B,C, io, m, n)$:
	\begin{enumerate}
		\item P: $(C, S) \leftarrow PC.Commit(pp, \widetilde{w})$ and send $C$ to V
		\item V: send $\tau \in^R \mathbb{F}^{log~m}$ to P
		\item let $T_1=0,~ \mu_1=log~m,~ l_1=3$
		\item V: set $r_x \in^R \mathbb{F}^{\mu_1}$
		\item Sum-check 1. $e_x \leftarrow <P_{SC}(G_{io,\tau}), V_{SC}(r_x)>(\mu_1, l_1, T_1)$
		\item P: compute $v_A=\overline{A}(r_x),~ v_B=\overline{B}(r_x),~ v_C=\overline{C}(r_x)$, send $(v_A, v_B, v_C)$ to V
		\item V: abort with $b=0$ if $e_x \neq (v_A \cdot v_B - v_C)\cdot \widetilde{eq}(r_x, \tau)$
		\item V: send $r_A, r_B, r_C \in^R \mathbb{F}$ to P
		\item let $T_2 = r_A \cdot v_A + r_B \cdot v_B + r_C \cdot v_C,~ \mu_2=log~m,~ l_2=2$
		\item V: set $r_y \in^R \mathbb{F}^{\mu_2}$
		\item Sum-check 2. $e_y \leftarrow <P_{SC}(M_{r_x}), V_{SC}(r_y)>(\mu_2, l_2, T_2)$
		\item P: $v \leftarrow \widetilde{w}(r_y[1..])$, send $v$ to V
		\item $b_e \leftarrow <P_{PC.Eval}(\widetilde{w}, S), V_{PC.Eval}(r)>(pp,  C, r_y, v, \mu_2)$
		\item V: abourt with $b=0$ if $b_e==0$
		\item V: $v_z \leftarrow (1 - r_y[0]) \cdot \widetilde{w}(r_y [1..]) + r_y[0] \widetilde{(io, 1)} (r_y[1..])$
		\item V: $v_1 \leftarrow \widetilde{A}(r_x, r_y),~ v_2 \leftarrow \widetilde{B}(r_x, r_y),~ v_3 \leftarrow \widetilde{C}(r_x, r_y)$
		\item V: abort with $b=0$ if $e_y \neq (r_A v_1 + r_B v_2 + r_C v_3) \cdot v_z$
		\item V: output $b=1$
	\end{enumerate}
\end{itemize}

\vspace{2cm}
\framebox{WIP: covered until sec.6}



\bibliography{paper-notes.bib}
\bibliographystyle{unsrt}

\end{document}
