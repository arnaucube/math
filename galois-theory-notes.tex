\documentclass{article}
\usepackage[utf8]{inputenc}
\usepackage{amsfonts}
\usepackage{amsthm}
\usepackage{amsmath}
\usepackage{enumerate}
\usepackage{hyperref}
\usepackage{amssymb}

\begin{filecontents}[overwrite]{galois-theory-notes.bib}
@misc{ianstewart,
  author = {Ian Stewart},
  title = {{Galois Theory, Third Edition}},
  year = {2004}
}

@misc{dihedral,
  author = {Gaurab Bardhan and Palash Nath and Himangshu Chakraborty}
  title = {Subgroups and normal subgroups of dihedral group up to isomorphism}
  year = {2010},
  note = {\url{https://scipp.ucsc.edu/~haber/ph251/Dn_subgroups.pdf}},
  url = {https://scipp.ucsc.edu/~haber/ph251/Dn_subgroups.pdf}
}
\end{filecontents}
\nocite{*}


\theoremstyle{definition}

\newtheorem{innerdefn}{Definition}
\newenvironment{defn}[1]
{\renewcommand\theinnerdefn{#1}\innerdefn}
{\endinnerdefn}

\newtheorem{innerthm}{Theorem}
\newenvironment{thm}[1]
{\renewcommand\theinnerthm{#1}\innerthm}
{\endinnerthm}

\newtheorem{innerlemma}{Lemma}
\newenvironment{lemma}[1]
{\renewcommand\theinnerlemma{#1}\innerlemma}
{\endinnerlemma}

\newtheorem{innercor}{Lemma}
\newenvironment{cor}[1]
{\renewcommand\theinnercor{#1}\innercor}
{\endinnercor}

\newtheorem{innereg}{Example}
\newenvironment{eg}[1]
{\renewcommand\theinnereg{#1}\innereg}
{\endinnereg}


\title{Galois Theory notes}
\author{arnaucube}
\date{2025}

\begin{document}

\maketitle

\begin{abstract}
	Notes taken while studying Galois Theory, mostyly from Ian Stewart's book "Galois Theory" \cite{ianstewart}.

	Usually while reading books and papers I take handwritten notes in a notebook, this document contains some of them re-written to $LaTeX$.

	The notes are not complete, don't include all the steps neither all the proofs.
\end{abstract}

\tableofcontents

\section{Recap on the degree of field extensions}
(Definitions, theorems, lemmas, corollaries and examples enumeration follows from Ian Stewart's book \cite{ianstewart}).

\begin{defn}{4.10}
  A \emph{simple extension} is $L:K$ such that $L=K(\alpha)$ for some $\alpha \in L$.
\end{defn}
\begin{eg}{4.11}
  Beware, $L=\mathbb{Q}(i, -i, \sqrt{5}, -\sqrt{5}) = \mathbb{Q}(i, \sqrt{5}) = \mathbb{Q}(i+\sqrt{5})$.
\end{eg}

\begin{defn}{5.5}
  Let $L:K$, suppose $\alpha \in L$ is algebraic over $K$. Then, the \emph{minimal polynomial} of $\alpha$ over $K$ is the unique monic polynomial $m$ over $K$, $m(t) \in K[t]$, of smallest degree such that $m(\alpha)=0$.
  \\
  eg.: $i \in \mathbb{C}$ is algebraic over $\mathbb{R}$. The minimal polynomial of $i$ over $\mathbb{R}$ is $m(t)=t^2 +1$, so that $m(i)=0$.
\end{defn}

\begin{lemma}{5.9}
  Every polynomial $a \in K[t]$ is congruent modulo $m$ to a unique polynomial of degree $< \delta m$.
\end{lemma}
\begin{proof}
  Divide $a / m$ with remainder, $a= qm +r$, with $q,r \in K[t]$ and $\delta r < \delta m$.
  Then, $a-r=qm$, so $a \equiv r \pmod{m}$.

  It remains to prove uniqueness.

  Suppose $\exists~ r \equiv s \pmod{m}$, with $\delta r, \delta s < \delta m$.
  Then, $r-s$ is divisible by $m$, but has smaller degree than $m$.

  Therefore, $r-s=0$, so $r=s$, proving uniqueness.
\end{proof}

\begin{lemma}{5.14}
  Let $K(\alpha):K$ be a simple algebraic extension, let $m$ be the minimal polynomial of $\alpha$ over $K$, let $\delta m =n$.

  Then $\{1, \alpha, \alpha^2, \ldots, \alpha^{n-1}\}$ is a basis for $K(\alpha)$ over $K$.
  In particular, $[K(\alpha):K]=n$.
\end{lemma}

\begin{defn}{6.2}
  The degree $[L:K]$ of a field extension $L:K$ is the dimension of L considered as a vector space over $K$.

  Equivalently, the dimension of $L$ as a vector space over $K$ is the number of terms in the expression for a general element of $L$ using coefficients from $K$.
\end{defn}

\begin{eg}{6.3}
  \begin{enumerate}
    \item $\mathbb{C}$ elements are 2-dimensional over $\mathbb{R}$ ($p+qi \in \mathbb{C}$, with $p,q \in \mathbb{R}$), because a basis is $\{1, i\}$, hence $[\mathbb{C}:\mathbb{R}]=2$.
    \item $[ \mathbb{Q}(i, \sqrt{5}) : \mathbb{Q}]=4$, since the elements $\{1, \sqrt{5}, i, i\sqrt{5}\}$ form a basis for $\mathbb{Q}(i, \sqrt{5})$ over $\mathbb{Q}$.
  \end{enumerate}
\end{eg}

\begin{thm}{6.4}\emph{(Short Tower Law)} \label{shorttowerlaw}
  If $K, L, M \subseteq \mathbb{C}$, and $K \subseteq L \subseteq M$, then $[M:K]=[M:L]\cdot [L:K]$.
\end{thm}
\begin{proof}
  Let $(x_i)_{i \in I}$ be a basis for $L$ over $K$,
  let $(y_j)_{j \in J}$ be a basis for $M$ over $L$.\\
  $\forall i \in I, j \in J$, we have $x_i \in L, u_j \in M$.
  \\
  Want to show that $(x_i y_j)_{i\in I, j\in J}$ is a basis for $M$ over $K$.
  \begin{enumerate}[i.]
    \item prove linear independence:\\
      Suppose that
      $$\sum_{ij} k_{ij} x_i y_j = 0 ~(k_{ij} \in K)$$
      rearrange
      $$\sum_j (\underbrace{\sum_i k_{ij} x_i}_{\in L}) y_j = 0 ~(k_{ij} \in K)$$
      Since $\sum_i k_{ij} x_i \in L$, and the $y_j \in M$ are linearly independent over $L$, then $\sum_i k_{ij} x_i = 0$.
      \\
      Repeating the argument inside $L$ $\longrightarrow$ $k_{ij}=0 ~~\forall i\in I, j\in J$.
      \\
      So the elements $x_i y_j$ are linearly independent over $K$.

    \item prove that $x_i y_j$ span $M$ over $K$:\\
      Any $x \in M$ can be written $x=\sum_j \lambda_j y_j$ for $\lambda_j \in L$, because $y_j$ spans $M$ over $L$.
      Similarly, $\forall j\in J,~ \lambda_j = \sum_i \lambda_{ij} x_i y_j$ for  $\lambda_{ij} \in K$.\\
      Putting the pieces together, $x=\sum_{ij} \lambda_{ij} x_i y_j$ as required.
  \end{enumerate}
\end{proof}

\begin{cor}{6.6}\emph{(Tower Law)}\\ \label{towerlaw}
  If $K_0 \subseteq K_1 \subseteq \ldots \subseteq K_n$ are subfields of $\mathbb{C}$, then
  $$[K_n:K_0] = [K_n:K_{n-1}] \cdot [K_{n-1}:K_{n-2}] \cdot \ldots \cdot [K_1: K_0]$$
\end{cor}
\begin{proof}
  From \ref{shorttowerlaw}.
\end{proof}

[...]


\newpage

\section{Tools}
This section contains tools that I found useful to solve Galois Theory related problems, and that don't appear in Stewart's book.

\subsection{De Moivre's Theorem and Euler's formula}\label{demoivre}
Useful for finding all the roots of a polynomial.

Euler's formula:
$$e^{i \psi} = cos \psi + i \cdot sin \psi$$

The n-th roots of a complex number $z=x + i y = r (cos \theta + i \cdot sin \theta)$ are given by

$$z_k = \sqrt[n]{r} \cdot \left(cos(\frac{\theta + 2k \pi}{n}) + i \cdot sin(\frac{\theta + 2k \pi}{n}) \right)$$
for $k=0, \ldots, n-1$.

So, by Euler's formula:
$$z_k = \sqrt[n]{r} \cdot e^{i (\frac{\theta + 2 k \pi}{n})}$$

\subsection{Einsenstein's Criterion} \label{einsenstein}
\emph{reference: Stewart's book}

Let $f(t) = a_0 + a_1 t + \ldots + a_n t^n$, suppose there is a prime $q$ such that
\begin{enumerate}
  \item $q \nmid a_n$
  \item $q | a_i$ for $i=0, \ldots, n-1$
  \item $q^2 \nmid a_0$
\end{enumerate}
Then, $f$ is irreducible over $\mathbb{Q}$.

\emph{TODO proof \& Gauss lemma.}


\subsection{Elementary symmetric polynomials}
\emph{TODO from orange notebook, page 36}

\subsection{Cyclotomic polynomials} \label{cyclotomicpoly}
\emph{TODO theory from brown muji notebook, page 82}

Examples:

\begin{align*}
  \Phi_n(x) &= x^{n-1} + x^{n-2} + \ldots + x^2 + x + 1 = \sum_{i=0}^{n-1} x^i\\
  \Phi_{2p}(x) &= x^{p-1} + \ldots + x^2 - x + 1 = \sum_{i=0}^{p-1} (-x)^i\\
  \Phi_m(x) &= x^{m/2} + 1, ~~\text{when $m$ is a power of $2$}
\end{align*}


\subsection{Lemma 1.42 from J.S.Milne's book}
\emph{TODO add reference to Milne's book}

Useful for when dealing with $x^p - 1$ with $p$ prime.

Observe that

$$x^p -1 = (x-1)(x^{p-1} + x^{p-2} + \ldots + 1)$$

Notice that
$$\Phi_p(x) = x^{p-1} + x^{p-2} + \ldots + 1$$
is the $p$-th Cyclotomic polynomial.

\begin{lemma}{1.42}
  If $p$ prime, then $x^{p-1} + \ldots + 1$ is irreducible; hence $\mathbb{Q}[e^{2 \pi i /p}]$ has degree $p-1$ over $\mathbb{Q}$.
\end{lemma}
\begin{proof}
  Let $f(x) = (x^p - 1)/(x-1) = x^{p-1} + \ldots + 1$
  then
  $$
  f(x+1) = \frac{(x+1)^p -1}{x+1-1} = \frac{(x+1)^p -1}{x} = x^{p-1} + \ldots + a_i x^i + \ldots + p
  $$

  with $a_i = \left( \stackrel{p}{i+1} \right)$.

  We know that $p | a_i$ for $i= 1, \ldots, p-2$, therefore $f(x+1)$ is irreducibe by Einsenstein's Criterion.

  This implies that $f(x)$ is irreducible.
\end{proof}


\subsection{Dihedral groups - Groups of symmetries} \label{dihedral}
Source: Wikipedia and \cite{dihedral}.

Dihedral groups ($\mathbb{D}_n$) represent the symmetries of a regular $n$-gon.

Properties:
\begin{itemize}
  \item are non-abelian (for $n>2$), ie. $rs \neq sr$
  \item order $2n$
  \item generated by a rotation $r$ and a reflextion $s$
  \item $r^n = s^2 = id,~~~(rs)^2=id$
\end{itemize}
Subgroups of $\mathbb{D}_n$:
\begin{itemize}
  \item rotation form a cyclic subgroup of order $n$, denoted as $<r>$
  \item for each $d$ such that $d|n$, $\exists~ \mathbb{D}_d$ with order $2d$
  \item normal subgroups
    \begin{itemize}
      \item for $n$ odd: $\mathbb{D}_n$ and $<r^d>$ for every $d|n$
      \item for $n$ even: $2$ additional normal subgroups
    \end{itemize}
  \item Klein four-groups: $\mathbb{Z}_2 \times \mathbb{Z}_2$, of order 4
\end{itemize}

\vspace{0.3cm}
Total number of subgroups in $\mathbb{D}_n$: $d(n) + s(n)$, where $d(n)$ is the number of positive disivors of $n$, and $s(n)$ is the sum of those divisors.

\begin{eg}{}
For $\mathbb{D}_6$, we have $\{1,2,3,6\} | 6$, so $d(n) = d(6) = 4$, and
$s(6) = 1+2+3+6 = 12$; henceforth, the total amount of subgroups is $d(n)+s(n) = 4+12 = 16$.
\end{eg}

\vspace{0.3cm}
For $n \geq 3, ~~\mathbb{D}_n \subseteq \mathbb{S}_n$ (subgroup of the Symmetry group).



\newpage

\section{Exercises}

\subsection{Galois groups}

\subsubsection[t6-7]{$t^6-7 \in \mathbb{Q}$}

This exercise comes from a combination of exercises 12.4 and 13.7 from \cite{ianstewart}.

First let's find the roots. By De Moivre's Theorem (\ref{demoivre}), $t_k =
\sqrt[6]{7} \cdot e^{i \frac{2 \pi k}{6}}$.

From which we denote $\alpha = \sqrt[6]{7}$, and $\zeta = e^{\frac{2 \pi i}{6}}$, so that the
roots of the polynomial are $\{ \alpha, \alpha \zeta, \alpha \zeta^2, \alpha \zeta^3, \alpha \zeta^4, \alpha \zeta^5\}$, ie. 
$\{ \alpha \zeta^k \}_0^5$.

Hence the \emph{splitting field} is $\mathbb{Q}(\alpha, \zeta)$.

\emph{Degree of the extension}

In order to find $[\mathbb{Q}(\alpha, \zeta) : \mathbb{Q}$, we're going to split it in tow
parts. By the Tower Law (\ref{towerlaw}),

$$[\mathbb{Q}(\alpha, \zeta) : \mathbb{Q}] = [\mathbb{Q}(\alpha, \zeta) : \mathbb{Q}(\alpha)] \cdot [\mathbb{Q}(\alpha) : \mathbb{Q}]$$

To find each degree, we will find the minimal polynomial of the adjoined term over the base field of the extension:

\begin{enumerate}[i.]
  \item minimal polynomial of $\alpha$ over $\mathbb{Q}$\\
    By Einsenstein's Criterion (\ref{einsenstein}), with $q=7$ we have that $q
    \nmid 1$, $7 | {-7,0,0,\ldots}$, and $7^2 \nmid -7$, hence $f(t)$ is
    irreducibe over $\mathbb{Q}$, thus is the minimal polynomial
    $$m_i(t)= f(t) =t^6-7$$
    which has roots $\{ \alpha \zeta^k \}_0^5$.
  \item minimal polynomial of $\zeta$ over $\mathbb{Q}(\alpha)$\\
    Since $\zeta$ is the primitive $6$th root of unity, we know that the minimal
    polynomial will be the $6$th cyclotomic polynomial (\ref{cyclotomicpoly}):
    $$m_{ii}(t) = \Phi_6(t) = t^2 - t + 1$$
    which has roots $\zeta, -\zeta$.

    Since $\mathbb{Q}(\alpha) \subseteq \mathbb{R}$, and the roots of
    $\Phi_6(t)=t^2 - t +1$ are in $\mathbb{C}$, $\Phi_6(t)$ remains irreducible
    over $\mathbb{Q}(\alpha)$.
\end{enumerate}

\vspace{0.5cm}
Therefore, by the tower of law,
$$[\mathbb{Q}(\alpha, \zeta) : \mathbb{Q}] = \deg{\Phi_6(t)} \cdot \deg{f(t)} = 2 \cdot 6 = 12$$
and by the Fundamental Theorem of Galois Theory, we know that
$$|\Gamma( \mathbb{Q}(\alpha, \zeta) : \mathbb{Q} )| = [\mathbb{Q}(\alpha, \zeta) : \mathbb{Q}] = 12$$
which tells us that there exist $12$ $\mathbb{Q}$-automorphisms of the Galois group.


\vspace{0.5cm}
Let's find the $12$ $\mathbb{Q}$-automorphisms. Start by defining $\sigma$ which
fixes $\zeta$ and acts on $\alpha$, sending it to another of the roots of the
minimal polynomial of $\alpha$ over $\mathbb{Q}$, $f(t)$, choose $\alpha \zeta$.

Now define $\tau$ which fixes $\alpha$ and acts on $\zeta$, sending it into
another root of the minimal polynomial of $\zeta$ over $\mathbb{Q}(\alpha)$,
choose $-\zeta$.

\vspace{0.3cm}
\begin{tabular}{@{}l l@{}}
    $\begin{aligned}
      \sigma: \alpha &\mapsto \alpha \zeta \\
      \zeta &\mapsto \zeta
    \end{aligned}$
  &
    $\begin{aligned}
      \tau: \alpha &\mapsto \alpha\\
      \zeta &\mapsto -\zeta = \zeta^{-1}
    \end{aligned}$
\end{tabular}

In other words, we have $12$ $\mathbb{Q}$-automorphisms, which are the
combination of $\sigma$ and $\tau$:

$$\begin{aligned}
  \sigma^k \tau^j:~~&\alpha \mapsto \alpha \zeta^k\\
		    &\zeta \mapsto \zeta^j
\end{aligned}$$

for $0 \leq k \leq 5$ and $j = \pm 1$.

\vspace{0.5cm}
\emph{TODO diagram}
\vspace{0.5cm}

Observe, that $\Gamma$ is generated by the combination of $\sigma$ and $\tau$,
and it is isomorphic to the group of symmetries of order 12, the dihedral
group (\ref{dihedral}) of order 12, $\mathbb{D}_6$, ie. $\Gamma \cong \mathbb{D}_6$.

\vspace{0.5cm}

Let's find the subgroups of $\Gamma$, and the fixed fields of $\mathbb{Q}(\alpha, \zeta)$.

We know that $\Gamma \cong \mathbb{D}_6$, and we know from the properties
of the dihedral group (\ref{dihedral}) that the number of subgroups of
$\mathbb{D}_6$ will be $d(6) + s(6) = 4 + 12 = 16$ subgroups.


\vspace{0.4cm}

\hspace*{-3.5cm}
\begin{tabular}{ c c c c | p{7.5cm} }
  \hline
  generators & order & group & fixed field & notes (check fixed field)\\
\hline
  $\langle \rangle = \langle \sigma^6 \rangle=\langle \tau^2 \rangle$ & 1 & id & $\mathbb{Q}(\alpha,\zeta)$ & \\
  $\langle \sigma \rangle = \langle \sigma^5 \rangle$ & 6 & $\mathbb{Z}_6$ & $\mathbb{Q}(\zeta)$ & \\
  $\langle \sigma^2 \rangle=\langle \sigma^4 \rangle$ & 3 & $\mathbb{Z}_3$ & $\mathbb{Q}(\alpha^3, \zeta)$ & $\sigma^2(\alpha^3)=\alpha^3 \zeta^{3\cdot 2}=\alpha^3 \zeta^6 = \alpha^3 \cdot 1 = \alpha^3$\\
  $\langle \sigma^3 \rangle$ & 2 & $\mathbb{Z}_2$ & $\mathbb{Q}(\alpha^2,\zeta)$ & $\sigma^3(\alpha^2)=(\alpha\zeta^3)^2=\alpha^2\zeta^6=\alpha^2$\\
  \hline
  $\langle \tau \rangle$ & 2 & $\mathbb{Z}_2$ & $\mathbb{Q}(\alpha)$ & \\
  \hline
  $\langle \sigma\tau \rangle$ & 2 & $\mathbb{Z}_2$ & $\mathbb{Q}(\alpha+\alpha\zeta)$ &
      $\sigma\zeta(\alpha+\alpha\zeta)=\sigma(\alpha+\alpha\zeta^{-1}) = \alpha\zeta + \alpha\zeta^{-1}\zeta=\alpha\zeta+\alpha$\\
  $\langle \sigma^2\tau \rangle$ & 2 & $\mathbb{Z}_2$ & $\mathbb{Q}(\alpha+\alpha\zeta^2), \mathbb{Q}(\alpha\zeta)$ &
      $\sigma^2\tau(\alpha+\alpha\zeta^2) = \sigma(\alpha+\alpha\zeta^{-2})=\alpha\zeta^2+ \alpha\zeta^{-2}\zeta^2=\alpha\zeta^2+\alpha$\\
  $\langle \sigma^3\tau \rangle$ & 2 & $\mathbb{Z}_2$ & $\mathbb{Q}(\alpha+\alpha\zeta^3)$ &
      $\sigma^3\tau(\alpha+\alpha\zeta^3) = \sigma(\alpha+\alpha\zeta^{-3})=\alpha\zeta^3+ \alpha\zeta^{-3}\zeta^3=\alpha\zeta^3+\alpha$\\
  $\langle \sigma^4\tau \rangle$ & 2 & $\mathbb{Z}_2$ & $\mathbb{Q}(\alpha+\alpha\zeta^4), \mathbb{Q}(\alpha\zeta^2)$ &
      $\sigma^4\tau(\alpha+\alpha\zeta^4) = \sigma(\alpha+\alpha\zeta^{-4})=\alpha\zeta^4+ \alpha\zeta^{-4}\zeta^4=\alpha\zeta^4+\alpha$\\
  $\langle \sigma^5\tau \rangle$ & 2 & $\mathbb{Z}_2$ & $\mathbb{Q}(\alpha+\alpha\zeta^5)$ &
      $\sigma^5\tau(\alpha+\alpha\zeta^5) = \sigma(\alpha+\alpha\zeta^{-5})=\alpha\zeta^5+ \alpha\zeta^{-5}\zeta^5=\alpha\zeta^5+\alpha$\\
  \hline
  $\langle \sigma, \tau \rangle = \langle \sigma^5,\tau \rangle$ & $6\cdot2=12$ & $\mathbb{D}_6$ & $\mathbb{Q}$ & \\
  $\langle \sigma^2, \tau \rangle = \langle \sigma^4,\tau \rangle$ & $3\cdot2=6$ & $\mathbb{D}_3$ & $\mathbb{Q}(\alpha^3)$ &
      $\sigma^2(\alpha^3)=\alpha^3\zeta^{3\cdot 2}=\alpha^3$ and $\tau(\alpha^3)=\alpha^3$\\
  $\langle \sigma^3, \tau \rangle$ & $2\cdot2=4$ & $\mathbb{D}_2$ & $\mathbb{Q}(\alpha^2)$ &
      $\sigma^3(\alpha^2)=\alpha^2\zeta^{2\cdot 2}=\alpha^2$ and $\tau(\alpha^2)=\alpha^2$\\
  \hline
  $\langle \sigma^2, \sigma\tau \rangle$ & $3\cdot 2=6$ & $\mathbb{D}_3$ & $\mathbb{Q}(\alpha^3+\alpha^3\zeta^3)$ &
      $\sigma^2(\alpha^3 + \alpha^3 \zeta^3) = \alpha^3\zeta^3 + \alpha^3 \zeta^3\zeta^3 = \alpha^3\zeta^3 + \alpha^3\zeta^6 = \alpha^3\zeta^3+\alpha^3$\\
  $\langle \sigma^3, \sigma\tau \rangle$ & $2\cdot2=4$ & $\mathbb{Z}_2 \times \mathbb{Z}_2$ & $\mathbb{Q}(\alpha^2\zeta^2),\mathbb{Q}(\alpha^2+\alpha^2\zeta^2)$ &
      $\sigma^3(\alpha^2+\alpha^2\zeta^2)=\alpha^2\zeta^{2\cdot3}+\alpha^2\zeta^{2\cdot3}\zeta^2=\alpha^2+\alpha^2\zeta^2$
      and
      $\sigma\tau(\alpha^2+\alpha^2\zeta^2)=\alpha^2\zeta^2+\alpha^2\zeta^{-2}\zeta^2 = \alpha^2\zeta^2+\alpha^2$\\
  $\langle \sigma^3, \sigma^2\tau\rangle$ & $2\cdot2=4$ & $\mathbb{Z}_2 \times \mathbb{Z}_2$ & $\mathbb{Q}(\alpha^2\zeta^4),\mathbb{Q}(\alpha^2+\alpha^2\zeta^4)$ &
      $\sigma^2\zeta(\alpha^2\zeta^4)=\alpha^2\zeta^2\zeta^{-4}=\alpha^2\zeta^{-2}=\alpha^2\zeta^4$
      and $\sigma^3(\alpha^2\zeta^4)=\alpha^2\zeta^{2\cdot3}\zeta^4=\alpha^2\zeta^4$
\end{tabular}



\bibliographystyle{unsrt}
\bibliography{galois-theory-notes.bib}

\end{document}
