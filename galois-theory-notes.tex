\documentclass{article}
\usepackage[utf8]{inputenc}
\usepackage{amsfonts}
\usepackage{amsthm}
\usepackage{amsmath}
\usepackage{enumerate}
\usepackage{hyperref}

\begin{filecontents}[overwrite]{galois-theory-notes.bib}
@misc{ianstewart,
  author = {Ian Stewart},
  title = {{Galois Theory, Third Edition}},
  year = {2004}
}
\end{filecontents}
\nocite{*}


\theoremstyle{definition}

\newtheorem{innerdefn}{Definition}
\newenvironment{defn}[1]
{\renewcommand\theinnerdefn{#1}\innerdefn}
{\endinnerdefn}

\newtheorem{innerthm}{Theorem}
\newenvironment{thm}[1]
{\renewcommand\theinnerthm{#1}\innerthm}
{\endinnerthm}

\newtheorem{innerlemma}{Lemma}
\newenvironment{lemma}[1]
{\renewcommand\theinnerlemma{#1}\innerlemma}
{\endinnerlemma}

\newtheorem{innercor}{Lemma}
\newenvironment{cor}[1]
{\renewcommand\theinnercor{#1}\innercor}
{\endinnercor}

\newtheorem{innereg}{Example}
\newenvironment{eg}[1]
{\renewcommand\theinnereg{#1}\innereg}
{\endinnereg}


\title{Galois Theory notes}
\author{arnaucube}
\date{2023-2024}

\begin{document}

\maketitle

\begin{abstract}
	Notes taken while studying Galois Theory, mostyly from Ian Stewart's book "Galois Theory" \cite{ianstewart}.

	Usually while reading books and papers I take handwritten notes in a notebook, this document contains some of them re-written to $LaTeX$.

	The notes are not complete, don't include all the steps neither all the proofs.
\end{abstract}

\tableofcontents

\section{Recap on the degree of field extensions}

\begin{defn}{4.10}
  A \emph{simple extension} is $L:K$ such that $L=K(\alpha)$ for some $\alpha \in L$.
\end{defn}
\begin{eg}{4.11}
  Beware, $L=\mathbb{Q}(i, -i, \sqrt{5}, -\sqrt{5}) = \mathbb{Q}(i, \sqrt{5}) = \mathbb{Q}(i+\sqrt{5})$.
\end{eg}

\begin{defn}{5.5}
  Let $L:K$, suppose $\alpha \in L$ is algebraic over $K$. Then, the \emph{minimal polynomial} of $\alpha$ over $K$ is the unique monic polynomial $m$ over $K$, $m(t) \in K[t]$, of smallest degree such that $m(\alpha)=0$.
  \\
  eg.: $i \in \mathbb{C}$ is algebraic over $\mathbb{R}$. The minimal polynomial of $i$ over $\mathbb{R}$ is $m(t)=t^2 +1$, so that $m(i)=0$.
\end{defn}

\begin{lemma}{5.9}
  Every polynomial $a \in K[t]$ is congruent modulo $m$ to a unique polynomial of degree $< \delta m$.
\end{lemma}
\begin{proof}
  Divide $a / m$ with remainder, $a= qm +r$, with $q,r \in K[t]$ and $\delta r < \delta m$.
  Then, $a-r=qm$, so $a \equiv r \pmod{m}$.

  It remains to prove uniqueness.

  Suppose $\exists~ r \equiv s \pmod{m}$, with $\delta r, \delta s < \delta m$.
  Then, $r-s$ is divisible by $m$, but has smaller degree than $m$.

  Therefore, $r-s=0$, so $r=s$, proving uniqueness.
\end{proof}

\begin{lemma}{5.14}
  Let $K(\alpha):K$ be a simple algebraic extension, let $m$ be the minimal polynomial of $\alpha$ over $K$, let $\delta m =n$.

  Then $\{1, \alpha, \alpha^2, \ldots, \alpha^{n-1}\}$ is a basis for $K(\alpha)$ over $K$.
  In particular, $[K(\alpha):K]=n$.
\end{lemma}

\begin{defn}{6.2}
  The degree $[L:K]$ of a field extension $L:K$ is the dimension of L considered as a vector space over $K$.

  Equivalently, the dimension of $L$ as a vector space over $K$ is the number of terms in the expression for a general element of $L$ using coefficients from $K$.
\end{defn}

\begin{eg}{6.3}
  \begin{enumerate}
    \item $\mathbb{C}$ elements are 2-dimensional over $\mathbb{R}$ ($p+qi \in \mathbb{C}$, with $p,q \in \mathbb{R}$), because a basis is $\{1, i\}$, hence $[\mathbb{C}:\mathbb{R}]=2$.
    \item $[ \mathbb{Q}(i, \sqrt{5}) : \mathbb{Q}]=4$, since the elements $\{1, \sqrt{5}, i, i\sqrt{5}\}$ form a basis for $\mathbb{Q}(i, \sqrt{5})$ over $\mathbb{Q}$.
  \end{enumerate}
\end{eg}

\begin{thm}{6.4}\emph{(Short Tower Law)}
  If $K, L, M \subseteq \mathbb{C}$, and $K \subseteq L \subseteq M$, then $[M:K]=[M:L]\cdot [L:K]$.
\end{thm}
\begin{proof}
  Let $(x_i)_{i \in I}$ be a basis for $L$ over $K$,
  let $(y_j)_{j \in J}$ be a basis for $M$ over $L$.\\
  $\forall i \in I, j \in J$, we have $x_i \in L, u_j \in M$.
  \\
  Want to show that $(x_i y_j)_{i\in I, j\in J}$ is a basis for $M$ over $K$.
  \begin{enumerate}[i.]
    \item prove linear independence:\\
      Suppose that
      $$\sum_{ij} k_{ij} x_i y_j = 0 ~(k_{ij} \in K)$$
      rearrange
      $$\sum_j (\underbrace{\sum_i k_{ij} x_i}_{\in L}) y_j = 0 ~(k_{ij} \in K)$$
      Since $\sum_i k_{ij} x_i \in L$, and the $y_j \in M$ are linearly independent over $L$, then $\sum_i k_{ij} x_i = 0$.
      \\
      Repeating the argument inside $L$ $\longrightarrow$ $k_{ij}=0 ~~\forall i\in I, j\in J$.
      \\
      So the elements $x_i y_j$ are linearly independent over $K$.

    \item prove that $x_i y_j$ span $M$ over $K$:\\
      Any $x \in M$ can be written $x=\sum_j \lambda_j y_j$ for $\lambda_j \in L$, because $y_j$ spans $M$ over $L$.
      Similarly, $\forall j\in J,~ \lambda_j = \sum_i \lambda_{ij} x_i y_j$ for  $\lambda_{ij} \in K$.\\
      Putting the pieces together, $x=\sum_{ij} \lambda_{ij} x_i y_j$ as required.
  \end{enumerate}
\end{proof}

\begin{cor}{6.6}\emph{(Tower Law)}\\
  If $K_0 \subseteq K_1 \subseteq \ldots \subseteq K_n$ are subfields of $\mathbb{C}$, then
  $$[K_n:K_0] = [K_n:K_{n-1}] \cdot [K_{n-1}:K_{n-2}] \cdot \ldots \cdot [K_1: K_0]$$
\end{cor}

\bibliographystyle{unsrt}
\bibliography{galois-theory-notes.bib}

\end{document}
