\documentclass{article}
\usepackage[utf8]{inputenc}
\usepackage{amsfonts}
\usepackage{amsthm}
\usepackage{amsmath}
\usepackage{enumerate}
\usepackage{hyperref}
\hypersetup{
    colorlinks,
    citecolor=black,
    filecolor=black,
    linkcolor=black,
    urlcolor=blue
}
\usepackage{xcolor}

% prevent warnings of underfull \hbox:
\usepackage{etoolbox}
\apptocmd{\sloppy}{\hbadness 4000\relax}{}{}

\theoremstyle{definition}
\newtheorem{definition}{Def}[section]
\newtheorem{theorem}[definition]{Thm}


\title{Notes on BLS Signatures}
\author{arnaucube}
\date{July 2022}

\begin{document}

\maketitle

\begin{abstract}
	Notes taken while reading about BLS signatures \cite{bls-sig-eth2}. Usually while reading papers I take handwritten notes, this document contains some of them re-written to $LaTeX$.

	The notes are not complete, don't include all the steps neither all the proofs.
\end{abstract}

% \tableofcontents

\section{BLS signatures}

\paragraph{Key generation}
$sk \in \mathbb{Z}_q$, $pk = [sk] \cdot g_1$, where $g_1 \in G_1$, and is the generator.

\paragraph{Signature}
$$\sigma = [sk] \cdot H(m)$$
where $H$ is a function that maps to a point in $G_2$. So $H(m), \sigma \in G_2$.

\paragraph{Verification}
$$e(g_1, \sigma) == e(pk, H(m))$$

Unfold:
$$e(pk, H(m)) = e([sk] \cdot g_1, H(m) = e(g_1, H(m))^{sk} = e(g_1, [sk] \cdot H(m)) = e(g_1, \sigma))$$

\paragraph{Aggregation}
Signatures aggregation:
$$\sigma_{aggr} = \sigma_1 + \sigma_2 + \ldots + \sigma_n$$
where $\sigma_{aggr} \in G_2$, and an aggregated signatures is indistinguishible from a non-aggregated signature.

\vspace{0.5cm}
Public keys aggregation:
$$pk_{aggr} = pk_1 + pk_2 + \ldots + pk_n$$
where $pk_{aggr} \in G_1$, and an aggregated public keys is indistinguishible from a non-aggregated public key.


\paragraph{Verification of aggregated signatures}
Identical to verification of a normal signature as long as we use the same corresponding aggregated public key:
$$e(g_1, \sigma_{aggr})==e(pk_{aggr}, H(m))$$

Unfold:
$$e(pk_{aggr}, H(m))= e(pk_1 + pk_2 + \ldots + pk_n, H(m)) =$$
$$=e([sk_1] \cdot g_1 + [sk_2] \cdot g_1 + \ldots + [sk_n] \cdot g_1, H(m))=$$
$$=e([sk_1 + sk_2 + \ldots + sk_n] \cdot g_1, H(m))=$$
$$=[sk_1 + sk_2 + \ldots + sk_n]~\cdot~e(g_1, H(m))=$$
$$=e(g_1, [sk_1 + sk_2 + \ldots + sk_n] \cdot H(m))=$$
$$=e(g_1, [sk_1] \cdot H(m) + [sk_2] \cdot H(m) + \ldots + [sk_n] \cdot H(m))=$$
$$=e(g_1, \sigma_1 + \sigma_2 + \ldots + \sigma_n)= e(g_1, \sigma_{aggr})$$


Note: in the current notes $pk \in G_1$ and $\sigma, H(m) \in G_2$, but we could use $\sigma, H(m) \in G_1$ and $pk \in G_2$.

\bibliography{paper-notes.bib}
\bibliographystyle{unsrt}

\end{document}
