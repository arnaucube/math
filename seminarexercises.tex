\documentclass{article}
\usepackage[utf8]{inputenc}
\usepackage{amsfonts}
\usepackage{amsthm}
\usepackage{amsmath}
\usepackage{enumerate}
\usepackage{hyperref}
\hypersetup{
	colorlinks,
	citecolor=black,
	filecolor=black,
	linkcolor=black,
	urlcolor=black
}

% custom solution environment to set custom numbers
\theoremstyle{definition}
\newtheorem{innersolution}{Solution}
\newenvironment{solution}[1]
{\renewcommand\theinnersolution{#1}\innersolution}
{\endinnersolution}

\title{Seminar exercises}
\author{ }
\date{February 2022}

\begin{document}
\maketitle

\begin{solution}{1.9}\
	\begin{enumerate}[1.]
		\item Let $f(a) = u$, then $g(f(a)) = g(u)$, so $g \circ f$ is a function.
		\item We can see that composition of functions is associative as follows:\\
			we know that $[ f \circ g](x) = f(g(x)), \forall x \in A$,\\
			so,
			$$(h \circ [g \circ f])(x) = h([g \circ f](x)) = h(g(f(x)))$$
			\\
			and
			$$([h \circ g] \circ f)(x) = [h \circ g](f(x)) = h(g(f(x)))$$
			Then, we can see that $$h \circ (g \circ f) = h(g(f(x))) = (h \circ g) \circ f$$
	\end{enumerate}
\end{solution}

\begin{solution}{1.28}\

	The quotient set of the equivalence relation in Example 1.27 is
	$$
	X / \sim = \{[(x_0,y_0)], [(x_1, y_1)], \ldots, [(x_n, y_n)]\}
	$$
	Yes, it is isomorphic to the cosets of the \emph{nth} roots of unity, which are $\mathbb{G}_n = \{w_k\}^{n-1}_{k=0}$, where $w_k=e^{\frac{2 \pi i}{n}}$.
\end{solution}

\begin{solution}{2.2}\

	To prove that the inverse $x^{-1}$ is unique, assume $x^{-1}$ and $\tilde{x}^{-1}$ are two inverses of $x$.\\
	By the definition of the inverse, we know that $x \cdot x^{-1} = e$. And by the definition of the unit element, we know that $x \cdot e = x$.\\
	Then, $$x^{-1} \cdot (x \cdot \tilde{x}^{-1}) = x^{-1} \cdot e = x^{-1}$$
	and $$(x^{-1} \cdot x) \cdot \tilde{x}^{-1} = e \cdot \tilde{x}^{-1} = \tilde{x}^{-1}$$
	By associativity property of groups, we know that
	$$x^{-1} \cdot (x \cdot \tilde{x}^{-1}) = (x^{-1} \cdot x) \cdot \tilde{x}^{-1}$$
	so, $$x^{-1} \cdot e = e \cdot \tilde{x}^{-1}$$
	which is $$x^{-1} = \tilde{x}^{-1}$$
	So, for any $x \in G$, the inverse $x^{-1}$ is unique.
\end{solution}

\begin{solution}{2.5}\

	Let $\alpha = (\begin{smallmatrix}1 & 2 & 3\\ 1 & 3 & 2\end{smallmatrix})$, $\beta = (\begin{smallmatrix}1 & 2 & 3\\ 3 & 1 & 2\end{smallmatrix})$, then,
	$$
	\alpha \cdot \beta = 
	(\begin{smallmatrix}1 & 2 & 3\\ 1 & 3 & 2\end{smallmatrix})
	\cdot (\begin{smallmatrix}1 & 2 & 3\\ 3 & 1 & 2\end{smallmatrix})
	= (\begin{smallmatrix}1 & 2 & 3\\ 3 & 2 & 1\end{smallmatrix})
	$$

	and
	$$
	\beta \cdot \alpha = 
	(\begin{smallmatrix}1 & 2 & 3\\ 3 & 1 & 2\end{smallmatrix}) \cdot
	(\begin{smallmatrix}1 & 2 & 3\\ 1 & 3 & 2\end{smallmatrix})
	= (\begin{smallmatrix}1 & 2 & 3\\ 2 & 1 & 3\end{smallmatrix})
	$$

	So, we can see that
	$$
	(\begin{smallmatrix}1 & 2 & 3\\ 3 & 2 & 1\end{smallmatrix})
	\neq
	(\begin{smallmatrix}1 & 2 & 3\\ 2 & 1 & 3\end{smallmatrix})
	$$

	so, $\alpha \cdot \beta \neq \beta \cdot \alpha$.
\end{solution}

\begin{solution}{2.26}\

	We want to prove that $f: G \rightarrow H$ is a \emph{monomorphism} iff $\ker f=\{e\}$.\\
	We know that $f$ is a \emph{monomorphism} (\emph{injective}) iff $\forall a, b \in G$, $f(a) = f(b) \Rightarrow a = b$.\\
	Let $a, b \in G$ such that $f(a)=f(b)$. Then
	$$f(a) f(b)^{-1} = f(b) (f(b))^{-1} = e$$
	$$f(a) f(b^{-1}) = e$$
	$$f(ab^{-1}) = e$$
	as $\ker f = \{e\}$, then we see that $ab^{-1}=e$, so $a=b$. Thus $f$ is a \emph{monomorphism}.
\end{solution}

\end{document}
